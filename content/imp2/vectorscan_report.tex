% SPDX-License-Identifier: LGPL-3-or-later
% Copyright 2023 VectorCamp
%
% Funded by NGI Search Programme HORIZON-CL4-2021-HUMAN-01 2022,
% https://www.ngisearch.eu/, EU Programme 101069364.

\chapter{Vectorscan}

\section{The Goal}

\href{https://github.com/VectorCamp/vectorscan}{Vectorscan} is a Portable Massively parallel Regular-Expression Matcher library.
It is used extensively in Intrusion Detection Systems software (like
\href{https://www.snort.org/}{Snort},
\href{https://suricata.io/}{Suricata} and others) and Network Security Analysis in general.
Originally the project was forked from Intel's \href{https://github.com/intel/hyperscan}{Hyperscan}
but attempts to port it to \acrshort{ARM} (originally) were not accepted upstream.
So a fork of Hyperscan was created in the form of Vectorscan, where portability is the main focus of the project.
\par

A few years later and Vectorscan is a popular project, with many external contributions.
It is currently heavily developed and continuously optimized and improved.
At the time of writing it is ported to \acrshort{ARM} and \acrshort{POWER} architectures.
Particularly for \acrshort{ARM}, Neon/\acrshort{ASIMD} support is 100\%
while there is ongoing work to port the code to \acrshort{SVE2} and the same Fat Runtime functionality as on x86 is implemented,
so that the same binary can run and take advantage of \acrshort{SVE2} if available.
\par

Furthermore, Loongson \acrshort{LSX} support is under review and we are in progress of porting it to even more architectures.
\par

Now, as part of this \acrshort{NGI} Search project, the goal is to port it to Libre-SOC and \acrshort{SVP64} in particular.
Unfortunately, \acrshort{SVP64} is a Vector architecture not a \acrshort{SIMD} one, and Vectorscan's whole codebase is designed around \acrshort{SIMD}.
\par

The whole codebase is tailored around \acrshort{SIMD} intrinsics and \acrshort{SIMD} data types to the point
that it is currently impossible to run it on an architecture that lacks a supported \acrshort{SIMD} unit.
\par

So, before actual \acrshort{SVP64} development on Vectorscan can even begin, we have to ensure two things:

\begin{enumerate}
  \item Vectorscan can run on a \acrshort{SIMD}-less architecture, without rewriting the whole code base.
  \item Vectorscan has to be adapted -at least partly at first- to make it easy for a single algorithm to be ported to \acrshort{SVP64}.
\end{enumerate}

\subsection{SIMD Everywhere in Vectorscan}

One possible solution would be to provide scalar implementations for every \acrshort{SIMD} intrinsic used in the project.
While this was certainly considered and some preliminary tests were done, we decided to opt for another solution.
\par

A very popular \acrshort{SIMD} library was used, \href{https://github.com/simd-everywhere/simde}{SIMD Everywhere/SIMDe}.
SIMDe is a header-only library that provides fast, portable implementations of \acrshort{SIMD} intrinsics on hardware which doesn't natively support them, such as calling \acrshort{SSE} functions on \acrshort{ARM}.
\par

A benefit of that approach is that it would allow Vectorscan to be not only portable to other architectures
for which no \acrshort{SIMD} support has been implemented yet, but also have a high performance.
In case \acrshort{SIMD} support for that architecture exists in SIMDe,
then Vectorscan would take advantage of that and perform better than with a scalar-only emulation of the \acrshort{SIMD} intrinsics.
\par

Finally, SIMDe provides an alternative backend of emulating the x86 intrinsics with native \acrshort{SIMD} intrinsics,
in order to compare performance of that and Vectorscan's own backend.
This will be especially useful in finetuning performance.
\par

A Pull Request has been made and merged to \href{https://github.com/VectorCamp/vectorscan/pull/203}{Vectorscan's Github}.
\par

Support for SIMDe is now officially in Vectorscan and it has been added as part of the
\href{https://buildbot-ci.vectorcamp.gr/#/grid}{Vectorscan CI pipeline}.
\par

\section{Vectorscan adaptation for \acrshort{SVP64}}

This is actually the harder part as it would need extensive refactoring of the whole codebase of Vectorscan.
Thankfully we do not have to do that, as after extensive analysis of the codebase we found that we can keep the same \acrshort{SIMD}-friendly algorithms
and slightly modify them so that they can accomodate the changes needed for predicates/masked loads.
We have adapted a single search engine from the many modules in Vectorscan.
The one we decided to work on is the already refactored Noodle search algorithm.
\par

Before we analyse the changes involved, let us first give an explanation of how the typical \acrshort{SIMD} algorithm
of processing a buffer of N bytes is implemented using the following pseudocode:

\begin{verbatim}
# Buffer is at memory address buf, size N
end = buf + N
IF N < SIMD_WIDTH THEN
   # PROCESS BYTES SCALAR
   RETURN
ENDIF

# Head:
IF ALIGNED(buf, SIMD_WIDTH) THEN
   # PROCESS BYTES UNTIL ALIGNED
   # INCREASE buf to aligned boundary
ENDIF

# Main loop: buf is now aligned to SIMD_WIDTH
# Now we can process using the full SIMD vectors
WHILE buf < end
   v = LOAD_VECTOR(buf)
   # PROCESS VECTOR v
   buf += SIMD_WIDTH
END WHILE

# Tail: Some bytes left
IF buf < end THEN
   # PROCESS BYTES SCALAR
ENDIF
\end{verbatim}

In some cases, the check for smaller sizes and the check for alignment can be merged, but the principle remains the same.
Now the benefit of a system like \acrshort{SVP64} is that you can skip the Head and Tail parts of the algorithm and just have a main loop.
However, the changes to a software like Vectorscan would have to be enormous, as it's completely tailored around \acrshort{SIMD} architectures.
However, \acrshort{SVP64} as well as at least two modern \acrshort{SIMD} architectures,
\acrshort{SVE}/\acrshort{SVE2} and \acrshort{AVX512} support the notion of predicated/masked loads.
In masked loads, we can use predicate masks to load a full vector on one hand until the aligned boundary,
but on the other hand only process the bytes that are not masked, and skip the rest.
\par

This has proven to be especially convenient and it allows a simple compromise between writing a fully optimized \acrshort{SVP64} algorithm
and providing something that will work with mostly the same code for all architectures.
We have to remember that for a project such as Vectorscan that supports as many architectures
and in the process of supporting even more, keeping things portable is of great importance
while at the same time it has to ensure that performance will not be severely impacted by the decisions made.
\par

We have to remember that for a project such as Vectorscan that supports as
many architectures and in the process of supporting even more, keeping
things portable is of great importance while at the same time it has
to ensure that performance will not be severely impacted
by the decisions made. \par

Fortunately Simple-V is designed with such portability in mind. Just like
with Cray Vectors, \acrshort{ARM} SVE/2, and \acrfull{RVV}, Scalable Vector Loops
may be written once, treating the assembler as if it was more
akin to an Architecturally-independent and stable "Software \acrshort{API}".
With Scalable Vectors, Programmers can be confident
that binary applications written years ago will not need any changes
to be faster or more power-efficient on future hardware. This is a very
different situation from \acrshort{SIMD} \acrshort{ISA}s which directly expose the
underlying \acrshort{SIMD} hardware to the programmer, where the routine addition
of new instructions to such \acrshort{SIMD} \acrshort{ISA}s is inherently guaranteed to require
a full assembly-level review and rewrite of the software. \par

Now that we have explained the changes involved, we will mention that we have changed Noodle to provide two separate code paths,
one for architectures that support masked loads (such as \acrshort{SVP64}, \acrshort{SVE2} and \acrshort{AVX512})
and the other for generic \acrshort{SIMD} architectures (like Neon, \acrshort{AVX2}, \acrshort{VSX}).
\par

The refactoring changes were too large to include here
but everything was done in \href{https://github.com/VectorCamp/vectorscan}{Vectorscan} and in particular as part of a large Pull Request:
\href{https://github.com/VectorCamp/vectorscan/pull/216}{Feature/refactor noodle masked loads \#216}
\par

We believe that these changes will make it easier to have an Noodle \acrshort{SVP64} implementation as part of \acrshort{IMP}3.
\par

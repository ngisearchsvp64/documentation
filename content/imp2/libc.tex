% SPDX-License-Identifier: LGPL-3-or-later
% Copyright 2023 VanTosh
%
% Funded by NGI Search Programme HORIZON-CL4-2021-HUMAN-01 2022,
% https://www.ngisearch.eu/, EU Programme 101069364.

\chapter{The choice of libc}

While we explored several C libraries, the idea of the project is to be as general as possible.
We explored several options in an effort to find the best available standard C library possible for usage within the scope of this project.
In this effort and in combination with other work that is relevant to the project, we ended up chosing ``glibc".
We provide an insight to several standard C libraries, we explored and tested with some feedback.
\par

\section{Newlib}
\label{sec:newlib}

This standard C library was quickly dismissed as support for any PowerPC or PowerISA would require adding support for newer instructions.
While this library is still in use by Google for some niche solutions, general support for this standard C library has dwindeled.
This standard C library was one of the basis, inspiration for Picolibc \ref{content:picolibc}.
\par

\section{AVR libc}
\label{sec:avrlibc}

AVR libc has no support for PowerPC or PowerISA processors, it targets mostly 8-bit \acrshort{RISC} processors.
This standard C library was one of the basis, inspiration for Picolibc \ref{content:picolibc}.

\section{Picolibc}
\label{content:picolibc}

This is a standard C library designed for smaller embedded devices with limited space.
Picolibc has been used on ARM based processors and on RISC-V process, recently PowerISA support was added.
Therefor a good candidate for use by SVP64, however it lacks good \acrfull{LE} support as it was based upon the PowerPC spec being \acrfull{BE}.
Adding full support for \acrfull{LE} is beyond the scope of this project and therefor this library was considered to be difficult to use.
We therefor did not perform any tests or practical usage of Picolibc.
\par

Another issue here is the lack of ecosystem or upstream usage, if we were to look at adding full support for Picolibc,
the eventual usage would still be limited to embedded devices, and not general purpose usage as there is no \acrfull{LSB} support.
\par

\section{dietlibc}

dietlibc is another standard C library designed for devices on a diet, again targeting embedded devices.
This standard library also support the PowerISA architecture, again from a PowerPC bases however it has full \acrfull{LE} support.
This means that adding SVP64 would be easier as most of the basic instruction are already present.
As a performance test, using the standard test suite, performed very poorly with string searches at round about 1 second.
In an exploration stage of looking to optimise these patterns and tests, we tried several changes to basic patters,
which turned out to reduce speed even more and this both on larger and smaller device bases.
After which this library was also considered out of scope for usage by our project.
\par

Again here, the lack of general support would mean a very niche solution that would never be able to run in general purpose situations.
\par

\section{klibc}
\label{content:klibc}

klibc is minimal standard C library targeted for use of the Linux Kernel startup process, as such it provided minimal features.
While it can be used in embedded devices, and MirBSD ksh is such an example the team has played with,
the general perception is that this standard C library is part of the Kernel development toolchain and expansion to this would be very difficult.
No real in-depth investigation was preformed beyond some basic checks.
\par

\section{Bionic}

Bionic is a standard C library used on Android, while the PowerISA support is not official by Google and not available under the open source project.
We evaluated the current open source implementation, however it lacks any possiblity to provide larger memory allocations
and will perform worse on larger memory devices,
As this standard C library takes a lot of inspiration from the BSD libc, by implementing OpenBSD and FreeBSD standard C concepts,
the project dismissed this standard C library.
\par

\section{uClibc \& uClibc-ng}
\label{content:uclibc}

This standard C library is designed for small devices, embedded devices and mobile devices, and has support for PowerPC based processors.
The support for newer PowerISA based processors is not complete and lacks several features,
for the intent of this project it would have been sufficient to complete string search.
However the basic performance tests ran into 1.5 seconds and did not support larger strings,
increasing the string size incurred into longer test runs and resulted into needing to complete several performance tests.
As the support for newer PowerISA is limited and the basic performance this standard C library was also considered out-of-scope.
\par

Unlike previous standard C library, this standard C library has some usage in real world applications such as
\begin{itemize}
\item
busybox
\item
OpenWRT
\item
Buildroot
\item
\ldots
\end{itemize}

\section{BSD libc}

The BSD libc standard C library is a default C library for OpenBSD, FreeBSD, NetBSD, and macOS.
As this library originates from the original 4.4BSD, it targets BSD based platforms.
For this project we are looking primarly to Linux based platforms as support for PowerISA on BSD is not at the same level.
However this standard C library is very efficient and fast, and was only dismissed due to the scope of this project.
\par

\section{musl}
\label{content:musl}

musl is a standard C library with clean efficient conformant implementations.
While at first glance it seemed very interesting and the easy, the perfomance was mediocare doing string search at 0.15 seconds.
The main problem was trying to get better performance as it required rewriting too much code and was deemed beyond the scope of this project.
While the test suites is easy to comprehend, it lacks larger subsets and more examples.
\par

This standard library is \acrshort{LSB} compliant for several architectures, however not for the PowerPC or PowerISA processors,
this would not pose a problem, however in light of several other factors as described above, this standard C library was not chosen.
\par

\section{glibc}
\label{content:glibc}

glibc is a standard C library designed for general purpose and it is currently the most used standard C library in open source projects,
as most Linux distributions provide glibc.
It has a full set of test suites and has good documentation, in addition to having project members have good knowledge of the standard C library.
Thie initial performance tests for string were one of the best, as these functions have already been tested against PowerISA based processors,
and they have been optimized over several interations.
Our initial tests had string search at less then 0.05 seconds on average.
The re-use possiblities are very high and the overhead of a ``larger" standard C library are quickly overlooked,
when performance and ease of usage are considered.
\par

This standard C library is \acrshort{LSB} compliant and supported by Debian, PowerEL, Fedora, RHEL, OpenSuSE, SLES, Gentoo
and several other Linux distributions of which PowerISA compliant processors are supported, including patched binutils to enable SVP64.
\par

% SPDX-License-Identifier: LGPL-3-or-later
% Copyright 2023 Red Semiconductor Ltd.
%
% Funded by NGI Search Programme HORIZON-CL4-2021-HUMAN-01 2022,
% https://www.ngisearch.eu/, EU Programme 101069364.

\chapter{Setting up the environment for working with glibc tests}

The setup script is available in the \texttt{documentation} repo
under \texttt{shell-scripts} directory.\\

Libre-SOC copy of the scripts will be made available under
\href{https://git.libre-soc.org/?p=dev-env-setup.git;a=tree}{dev-env-setup}.

\section{Setting up the Libre-SOC chroot}

Create LibreSOC chroot (host system):

\begin{verbatim}
#: cd /PATH/TO/dev-env-setup
#: ./mk-deb-chroot glibc-svp64
#: ./cp-scripts-to-chroot glibc-svp64
\end{verbatim}

Scripts inside chroot:

\begin{verbatim}
$: schroot -c glibc-svp64
(glibc-svp64)$: cd ~/dev-env-setup
(glibc-svp64)$: sudo bash
(glibc-svp64)#: ./install-hdl-apt-reqs
(glibc-svp64)#: ./binutils-gdb-install
(glibc-svp64)#: ./hdl-dev-repos
(glibc-svp64)#: exit
(glibc-svp64)$: cd ~/src
(glibc-svp64)$: git clone https://github.com/ngisearchsvp64/shell-scripts.git
(glibc-svp64)$: cd shell-scripts/scripts
(glibc-svp64)$: ./ngi-search-glibc-svp64
\end{verbatim}

Each function test regression can be run by calling:

\begin{verbatim}
(glibc-svp64)$: cd ~/src/glibc-svp64/svp64-port
(glibc-svp64)$: ./SILENCELOG='!instr_in_outs' ./test-memchr-svp64 --direct
\end{verbatim}

For no simulator log, set \texttt{SILENCELOG=1}.
For full logging, skip setting \texttt{SILENCELOG}.

The full test regression may take several days (depending on your hardware).
So far, some tests fails. Some of the failures are due to a manually reduced
maximum string size used in the test (which was done for reducing the time
taken to run regressions).\\

To save the regression results and simulator log:

\begin{verbatim}
(glibc-svp64)$: ./test-memchr-svp64 --direct >& /tmp/f
\end{verbatim}

To check for failed tests, run:

\begin{verbatim}
(glibc-svp64)$: grep "Wrong" /tmp/f
\end{verbatim}

\section{Generate this document}

This document is written in \LaTeX{} and a PDF can be generated using provided
shell script. The script works with the LibreSOC environment, however the
actual \LaTeX{} document can be generated on any system.

\begin{verbatim}
(glibc-svp64)$: cd ~/src/shell-scripts/scripts
(glibc-svp64)$: ./ngi-search-docs
\end{verbatim}

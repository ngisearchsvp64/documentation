% SPDX-License-Identifier: LGPL-3-or-later
% Copyright 2023 VectorCamp
% Copyright 2023 Red Semiconductor Ltd.
% Copyright 2023 VanTosh
%
% Funded by NGI Search Programme HORIZON-CL4-2021-HUMAN-01 2022,
% https://www.ngisearch.eu/, EU Programme 101069364.

\section{\texttt{memchr} Function Breakdown}

\subsection{What is \texttt{memchr}?}

\begin{verbatim}
void *memchr(const void *s, int c, size_t n);
\end{verbatim}

The function \texttt{memchr} is part of the standard C library, and can be used
after adding an \texttt{\#include <string.h>}.

\begin{verbatim}
The memchr function locates the first occurrence of c
(converted to an unsigned char) in the initial n characters
(each interpreted as unsigned char) of the object pointed to by s.
The implementation shall behave as if it reads the characters sequentially
and stops as soon as a matching character is found.

The memchr function returns a pointer to the located character,
or a null pointer if the character does not occur in the object.
\end{verbatim}

To summarise: find the first occurance of unsigned char \texttt{c} in the
array of chars pointed to by \texttt{s}. No match returns a null.

\subsection{\texttt{memchr\_wrapper.c} - C wrapper to interface glibc test with SVP64 code}

Source code for \texttt{memchr\_wrapper.c} can be found in the
\texttt{glibc-svp64} repo, path is:
\begin{verbatim}
svp64-port/memchr_wrapper.c
\end{verbatim}

This wrapper file defines a wrapper function \texttt{MEMCHR\_SVP64()} with the
same input arguments as standard \texttt{memchr()}, but also includes
register setup and call to \texttt{pypowersim} which will run
LibreSOC's \texttt{ISACaller} simulator only for
the \acrshort{SVP64} \texttt{memchr} implementation.\\

The string to be tested is copied from the native C address space, to the
simulator's memory area (this is why during the execution of \texttt{memchr},
a different string address will be seen, although the offsets within the
string are the same).

\subsection{\texttt{test-memchr.c} - Test regressions for \texttt{memchr}}

Source code for \texttt{test-memchr.c} can be found in the
\texttt{glibc-svp64} repo, path is:
\begin{verbatim}
svp64-port/test-memchr.c
\end{verbatim}

This file is copied from glibc, and requires several modifications to support
the \acrshort{SVP64} implementation. See Section \ref{sec:adding_new_func},
page \pageref{sec:adding_new_func}

\subsection{\texttt{memchr\_svp64.s} - \acrshort{SVP64} assembler implementation of \texttt{memchr}}

Source code for \texttt{memchr\_svp64.s} can be found in the
\texttt{glibc-svp64} repo, path is:
\begin{verbatim}
svp64-port/svp64/memchr_svp64.s
\end{verbatim}

\subsubsection{Setup}

Setup variable names (set symbol) corresponding to \acrfull{GPR} which are used
during the algorithm.

\begin{itemize}
  \item Convience register labels: \texttt{in\_ptr}, \texttt{c}, \texttt{n},
  \texttt{tmp}, \texttt{ctr}, \texttt{s}, \texttt{t}.
\end{itemize}

A macro is defined which takes the character to match, \texttt{c},
and produces an 8-byte mask with 8 copies of \texttt{c}.
This allows to compare 8 bytes of the input string at a time
(when loading 8-bytes from memory).

\begin{itemize}
  \item Example: `c=0x17`, macro will produce the mask `0x1717171717171717`.
\end{itemize}

\subsubsection{Outer loop}

\begin{enumerate}
  \item Must determine if number of remaining chars, \texttt{n} of
  string is zero. If so, return null (no match found).
  \item If \texttt{n} is less than 32, proceed to the byte search part of
  the algorithm (\texttt{.found}).
  \item If \texttt{n} is 32 or greater, can use the SVP64 Vertical-First
  routine, \texttt{.inner}. If this is the case, do the preamble
  before \texttt{.inner}:
  \begin{enumerate}
    \item First store value of \acrshort{CTR} in \texttt{ctr}.
    Configure the \acrshort{CTR}. \acrshort{CTR} is used to keep
    track of loop iterations (this is standard PowerISA use,
    not specific to SVP64). \acrshort{CTR} is also used for
    certain conditional branch modes.
    \item Setup \acrshort{SVSTATE} register using \texttt{setvl}.
    \acrfull{VL} is 4 (value in \texttt{ctr}), \acrfull{VF} mode is used
    (step over instructions before
    incrementing \texttt{srcstep}, \texttt{dststep} to the next element).
    For more detailed info on \acrshort{VF},
    \href{https://libre-soc.org/openpower/sv/svstep/}{see wiki page}.
  \end{enumerate}
\end{enumerate}

\subsubsection{Inner loop}

Comprises the inner loop of the \acrshort{SVP64} Vertical-First algorithm.

\begin{itemize}
  \item Load doubleword (8 bytes) of the string from memory and store in 
register starting at \texttt{s} (actual register used is \texttt{16+dststep}).
  \item Compare byte by byte the loaded 8 bytes of the string with the matching
char mask. If any of the 8 bytes match, the instruction will set
the corresponding byte to \texttt{0xff} in \texttt{t}
(starts at \texttt{GPR[32]}).
\end{itemize}

Example:
\begin{verbatim}
call cmpb cmpb
read reg 16/0: 0x9715a432c157d17
read reg 8/0: 0x1717171717171717
read reg 32/0: 0x0
write reg r(32, 32, 0) 0xff ew 64
\end{verbatim}

\begin{itemize}
  \item Compare immediate value 0 with \texttt{t}. Set the \acrshort{CR} Field
  (starting at \acrshort{CR} Field 0). If a matching char is present in the 8-byte
  segment of the string, \texttt{t} will have a non-zero value, thus the
  \acrshort{CR} Field EQ bit will be cleared. This bit will be used to determine whether
  to branch to the end of the algorithm, or to continue.
  \item Vectorised branch conditional. In the event of a matching byte,
  EQ bit of \acrshort{CR} Field will be zero, and thus a branch to \texttt{.found}
  will occur if \acrshort{CTR} is also non-zero.
  \item If no matches have occurred, continue by incrementing element using
  the \texttt{svstep} instruction. More info
  \href{https://libre-soc.org/openpower/sv/svstep/}{here}.
  \item Increment address stored in \texttt{in\_ptr} by 8, as the code checked
  8 bytes of string, and now can continue on to the next portion of the string.
  Decrement \texttt{n} by 8 as there are now 8 fewer bytes to check.
  \item Branch back to the start of \texttt{.inner} if \acrshort{CTR} is non-zero
  (haven't checked all 32 bytes yet). Otherwise branch back to \texttt{.outer}
  to determine how to continue.
\end{itemize}

\subsubsection{Found loop}

This code is reached if:

\begin{itemize}
  \item There are fewer than 32 bytes to process, or,
  \item A match has been found in \texttt{.inner} and need to determine the
  exact byte address
\end{itemize}

The algorithm is similar to the one used in \texttt{.inner}, except that code tests
one byte at a time. If there's an exact match, then the code returns from
\texttt{memchr} function call.

\subsubsection{Tail}

If this portion of code is reached (or jumped to), then no
match has been found. Null is returned.

\subsubsection{Post-analysis issues discovered}

\begin{itemize}
  \item \acrshort{CTR} register is decremented twice in the \texttt{inner} loop, once by
  \texttt{sv.bc}, and again by \texttt{bdnz}.
  This means the number of iterations is only 2 (as opposed to 4).
  \item \texttt{svstep} third argument needs to be set to 1 to enable
  element incrementing. Enabling this however, caused a strange issue where the
  \acrfull{CTR} used in \texttt{sv.ld} was not being changed even
  after the value stored in \texttt{in\_ptr} was incremented by 8
  (\acrshort{EA} was effectively frozen for the duration of Vertical-First).
\end{itemize}

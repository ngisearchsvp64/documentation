% SPDX-License-Identifier: LGPL-3-or-later
% Copyright 2023 Red Semiconductor Ltd.
%
% Funded by NGI Search Programme HORIZON-CL4-2021-HUMAN-01 2022,
% https://www.ngisearch.eu/, EU Programme 101069364.

\chapter{Guide to New SVP64 Implementation of a glibc Function}
\label{sec:adding_new_func}

The directory of the glibc repo is the same as for initial environment setup:

\begin{verbatim}
(glibc-svp64)$: cd ~/src/glibc-svp64/glibc
\end{verbatim}

\section{Required steps - \texttt{strchr} as an example}

\subsection{Copy \texttt{test-[function].c} from glibc}

From glibc, copy the test C file for the respective function
into \texttt{svp64-port} directory.

It should be under the function category, for example, strchr is
under \texttt{string/test-strchr.c}:

\begin{verbatim}
(glibc-svp64)$: cp ~/src/glibc-svp64/glibc/string/test-strchr.c \
                ~/src/glibc-svp64/svp64-port/
\end{verbatim}

\subsection{Making adjustments to \texttt{test-[function].c}}

Add the following \texttt{\#define}'s after the
\texttt{typedef CHAR *(*proto\_t) (const CHAR *, int);} line:

\begin{verbatim}
#define STRCHR_SVP64 strchr_svp64
#define MAX_SIZE    256
CHAR *SIMPLE_STRCHR (const CHAR *, int, size_t);
CHAR *STRCHR_SVP64 (const CHAR *, int, size_t);
\end{verbatim}

\texttt{MAX\_SIZE} is set to 256 to reduce the time taken to run
the regression tests.\\

The \texttt{IMPL} calls will need to be modified as they define the
functions used during the test. There was an inconsistency with the
upper/lower case \texttt{simple\_} prefix, but that might just be a
historical reason.\\

Comment out:

\begin{verbatim}
IMPL (stupid_STRCHR, 0)
IMPL (simple_STRCHR, 0)
IMPL (STRCHR, 1)
\end{verbatim}

And add:

\begin{verbatim}
IMPL (STRCHR_SVP64, 1)
IMPL (simple_STRCHR, 2)
\end{verbatim}

For debugging purposes, add a printf statement as first line inside
\texttt{SIMPLE\_[FUNCTION]} function code.
For \texttt{strchr} the statement is added inside \texttt{SIMPLE\_STRCHR}:

\begin{verbatim}
printf("strchr called: s: %p, c: %02x(%c)\n", s, (uint8_t)c, c);
\end{verbatim}

For initial testing, it's worthwhile to disable most tests,
and only turn on a few. The C function \texttt{test\_main} at the bottom
of the file contains the tests being run.\\

\texttt{do\_test} is the function used for running a single test case, and
has five arguments (for the \texttt{strchr)}:

\begin{itemize}
  \item \texttt{align} - byte alignment in the array
  \item \texttt{pos} - position in the array
  \item \texttt{len} - length of char array
  \item \texttt{seek\_char} - when \texttt{align} and \texttt{pos} set to 0,
  this is equal to the character the function will search for.
  \item \texttt{max\_char} - Largest permitted character (buffer char is
  limited by performing modulo \texttt{max\_char}).
\end{itemize}

In the test cases done for \texttt{memchr}, \texttt{memrchr}, \texttt{strchr},
the length argument is limited to 256 for reducing the time take to run tests.\\

Inside the \texttt{SIMPLE\_[FUNCTION]} function, add the printf statement
for debug and/or logging:

\begin{verbatim}
printf("strchr called: s: %p, c: %02x(%c)\n", s, (uint8_t)c, c);
\end{verbatim}

\subsection{\texttt{[function]\_wrapper.c}}

Create a new \texttt{strchr\_wrapper.c} C file which will interface with
the glibc tests and access the ISACaller PowerISA+SVP64 simulator. An existing
\texttt{[function]\_wrapper.c} file can be used (in this example,
\texttt{memchr}):

\begin{verbatim}
(glibc-svp64)$: cp ~/src/glibc-svp64/svp64-port/memchr_wrapper.c \\
                ~/src/glibc-svp64/svp64-port/strchr_wrapper.c
\end{verbatim}

\subsection{Making adjustments to \texttt{[function]\_wrapper.c}}

Sadly this is difficult to automate (at least for now), because other than
substituting the function name, the logic of the code may need to change.
The general structure will however remain the same.

\begin{itemize}
  \item The easiest change to make is to replace every instance of previous
  function name to the new one being implemented. With the current example,
  replace \texttt{memchr} with \texttt{strchr},
  and \texttt{MEMCHR} with \texttt{STRCHR}.
  \item Change the input arguments of \texttt{[FUNCTION]\_SVP64}.
  In this case, only \texttt{s} and \texttt{c} are necessary
  (as string function continues until a null byte is encountered.
  \item Update the code and comments of the \texttt{[FUNCTION]\_SVP64}.
  In this case, remove use of \texttt{n}. \texttt{size\_t bytes} need to be
  calculated using \texttt{strlen(s)} because the length of the string
  is not provided.
  \item Make sure to update the copyright notice at the top.
\end{itemize}

\subsection{Adjusting Makefile}

The Makefile in \texttt{svp64-port} needs to be updated to include the new
function test code (in this example, \texttt{strchr}).

Add a new target:

\begin{verbatim}
strchr_TARGET	= test-strchr-svp64
\end{verbatim}

Below \texttt{BINDIR} variable add:
\begin{verbatim}
strchr_CFILES	:= support_test_main.c test-strchr.c strchr_wrapper.c
strchr_ASFILES := $(SVP64)/strchr_svp64.s $(SVP64)/strchr_orig_ppc64.s
strchr_SVP64OBJECTS := $(strchr_ASFILES:$(SVP64)/%.s=$(SVP64)/%.o)
strchr_OBJECTS := $(strchr_CFILES:%.c=%.o)
strchr_BINFILES := $(BINDIR)/strchr_svp64.bin
strchr_ELFFILES := $(BINDIR)/strchr_svp64.elf
\end{verbatim}

Add target for the \texttt{test-[function].o} object:
\begin{verbatim}
test-strchr.o: test-strchr.c
	$(CC) -c $(CFLAGS) -DMODULE_NAME=testsuite -o $@ $^
\end{verbatim}

Add target for generating assembly implementation of using standard PowerISA:
\begin{verbatim}
$(SVP64)/strchr_orig_ppc64.s: $(GLIBCDIR)/string/strchr.c
	$(CROSSCC) $(CROSSCFLAGS) -S -g0 -Os -DMODULE_NAME=libc -o $@ $^
\end{verbatim}

Append \texttt{\$(strchr\_TARGET)} to the \texttt{all} make rule.

Add a target for the final \texttt{test-[function]-svp64} binary:
\begin{verbatim}
$(strchr_TARGET): $(strchr_OBJECTS) $(strchr_SVP64OBJECTS) $(strchr_ELFFILES) \
$(strchr_BINFILES)
	$(CC) -o $@ $(strchr_OBJECTS) $(LDFLAGS)
\end{verbatim}

Add a line to the \texttt{clean} make rule:
\begin{verbatim}
$ rm -f $(strchr_OBJECTS) $(strchr_SVP64OBJECTS) $(strchr_BINFILES) \
$(strchr_ELFFILES) $(strchr_TARGET)
\end{verbatim}

Append \texttt{\$(strchr\_TARGET)} to the line under the \texttt{remove} make rule.

\subsection{\texttt{[function]\_svp64.s} assembler file}

This file can be started by copying from existing SVP64 function, or by using
the generated assembler using the reference implementation
(\texttt{[function]\_orig\_ppc64.s}), although the generated assembler is
probably more difficult to follow than simply writing from scratch.\\

If copying from the \texttt{memchr} SVP64 assembler:
\begin{verbatim}
(glibc-svp64)$: cp ~/src/glibc-svp64/svp64-port/svp64/memchr_svp64.s \\
                ~/src/glibc-svp64/svp64-port/svp64/strchr_svp64.s
\end{verbatim}

Modifications needed to be made:
\begin{itemize}
  \item Change \texttt{memchr} string to \texttt{strchr}.
  \item Make modifications based on the operation of the new function
  (and difference in input arguments).
\end{itemize}

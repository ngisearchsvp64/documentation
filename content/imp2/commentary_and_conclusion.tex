\chapter{Commentary and Conclusions}

VectorCamp, VanTosh have been \textit{users} and \textbf{not} developers
of Simple-V during this task.
Therefore the goal during this milestone was to use the capabilities of a
frozen version of Simple-V in order to provide the research and
groundwork for the next milestone.

The initial code produced will then be worked and improved upon in milestone 3
as the implemented wrapper now allows for the integration with the
\texttt{glibc} test framework.

The decision to use binutils allowed to use standard features of
PowerISA assembler (such as instruction aliases).
However, the use of the older version of binutils imposed limitations
on the number of features of
Simple-V which could be used. This is why, for example,
Data-Dependent Fail-First was not used, and hence why the size of
\texttt{memchr}, \texttt{memrchr}, and \texttt{strchr} were
substantially larger than theoretically possible.

The implemented functions, \texttt{memchr}, \texttt{memrchr},
and \texttt{strchr}, demonstrate that even a subset of Simple-V can be used
to implement standard C library string functions \textit{as well as} be
integrated into \texttt{glibc} testing framework
(see Section \ref{sec:debug_run_tests}, page \pageref{sec:debug_run_tests}).

For the purpose of this milestone, VectorCamp fixed the version of
\texttt{openpower-isa} to a particular commit,
\href{https://git.libre-soc.org/?p=openpower-isa.git;a=commitdiff;h=d9544764b1710f3807a9c0685d150a665f70b9a2}{commit \#d9544764}.
This meant it was not possible to take advantage of an existing fix
with binutils address calculation (manual calculation covered in Section
\ref{subsubsec:wrong_val_man_addr},
page \pageref{subsubsec:wrong_val_man_addr}).

\chapter{Commentary and Conclusions}

VectorCamp, VanTosh have been \textit{users} and \textbf{not} developers
of Simple-V, and have been supported by Red Semiconductor during this task.
Therefore the goal during this milestone was to push the capabilities of a
frozen version of Simple-V, and not to try to change
and/or add new features. \\

Due to the decision of using binutils which allowed to use standard features of
PowerISA assembler, limitations were imposed on the number of features of
Simple-V which could be used. This is why, for example,
Data-Dependent Fail-First was not available, and hence why the size of
\texttt{memchr}, \texttt{memrchr}, and \texttt{strchr} were
substantially larger than theoretically possible.
The lagging support of features in binutils does not in any way imply that
Simple-V has limited potential for code size reduction and clarity.\\

The implemented functions, \texttt{memchr}, \texttt{memrchr},
and \texttt{strchr}, demonstrate that even the subset of Simple-V can be used
to implement standard C library string functions \textit{as well as} be
integrated into \texttt{glibc} testing framework
(see Section \ref{sec:debug_run_tests}, page \pageref{sec:debug_run_tests}).

Because for the purpose of this milestone, we fixed our version of
\texttt{openpower-isa} to a particular commit,
\href{https://git.libre-soc.org/?p=openpower-isa.git;a=commitdiff;h=d9544764b1710f3807a9c0685d150a665f70b9a2}{commit \#d9544764},
we were not able to take advantage of an existing fix with binutils
address calculation (manual calculation covered in Section
\ref{subsubsec:wrong_val_man_addr},
page \pageref{subsubsec:wrong_val_man_addr}).
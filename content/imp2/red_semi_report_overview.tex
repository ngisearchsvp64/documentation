\section{Red Semiconductor Report Overview}

\subsection{Red's goals}

\subsubsection{Improving Simple-V spec}

\begin{enumerate}
  \item To check and improve LibreSOC's current Simple-V spec by finding/implementing
  new features relevant to processing strings and search algorithms in general
  (as part of NGI Search grant goals).
\end{enumerate}

An example feature which came out of this research is
Data-Dependent Fail-First, which allows to iterate over data (in
this context, an array of byte characters) and terminate early when
a condition is met (hence "fail-first"), and shorten the specified
vector length based on this early termination. This feature is very useful
for string search and copy functions. One great example is the \texttt{strncpy}
function, code \href{https://git.libre-soc.org/?p=openpower-isa.git;a=blob;f=src/openpower/decoder/isa/test_caller_svp64_ldst.py;h=4ecf534777a5e8a0178b29dbcd69a1a5e2dd14d6;hb=d9544764b1710f3807a9c0685d150a665f70b9a2#l31}{here}
\footnote{LibreSOC openpower commit \texttt{\#d9544764...}
(\texttt{fix vertical-first sv.bc}) is used to refer to
SVP64 spec and code examples. Commit link
\href{https://git.libre-soc.org/?p=openpower-isa.git;a=commit;h=d9544764b1710f3807a9c0685d150a665f70b9a2}{here}.}

\subsubsection{Simple-V assistance to VectorCamp and Vantosh}


\begin{enumerate}
  \item Provide assistance to VectorCamp and Vantosh with currently implemented
features of Simple-V.
  \item To find causes and/or workarounds with code bugs due to the ISACaller
simulator (the simulator supporting PowerISA 3.0 and implements current Simple-V
specification).
  \begin{enumerate}
    \item If a bug related to the simulator is found, to create a test
  case and submit a bug report to LibreSOC (see the \texttt{sv.bc} branching address
  bug in \href{https://bugs.libre-soc.org/show_bug.cgi?id=1210}{bug \#1210}
  as an example).
    \item Current workarounds required due to binutils support lagging behind
  current Simple-V (due to time and effort constraints in LibreSOC):
    \begin{enumerate}
      \item  Use original assembler instruction instead of extended mnemonics
      when using SimpleV \texttt{sv.} prefix. Examples of extended mnemonics
      are \texttt{nop}, \texttt{bdnz}, \texttt{subi}, etc.). Instead have to use 
      \texttt{ori 0,0,0}, \texttt{bc 16,0,target}, \texttt{addi Rx,Ry,-value}.
    \end{enumerate}
    \item \texttt{sv.bc} with \texttt{BO=0} (other \texttt{BO} modes weren't tested, see
  Figure 40. BO field encodings in section 2.4 of PowerISA 3.0 Book I
  for the full list), has to be given a manually computed address.
  See an example of how to this LINK TO "debugging function test"??
  \end{enumerate}
\end{enumerate}

\subsubsection{String glibc function utilising subset of SimpleV feature set}

To write implementations of several glibc string functions utilising a subset
of the available SimpleV feature set

\subsubsection{Integration into the glibc test environment}

VectorCamp has done the initial work of studying glibc,
creating wrapper code, and writing up an example function, \texttt{memchr()},
for Vantosh and Red to use as a template (from which the other functions were
written up).

\subsubsection{Documentation}

\subsubsection{Environment setup}

\subsubsection{Running existing test regressions}

\subsubsection{Creating new function implementations}

\subsubsection{Debugging function when failures occur}

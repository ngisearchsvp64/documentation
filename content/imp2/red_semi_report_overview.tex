\section{Red Semiconductor Report Overview}

\subsection{Red's goals}

\subsubsection{Improving Simple-V spec}

\begin{enumerate}
  \item To check and improve LibreSOC's current Simple-V spec by finding/implementing
  new features relevant to processing strings and search algorithms in general
  (as part of NGI Search grant goals).
\end{enumerate}

An example feature which came out of this research is
Data-Dependent Fail-First, which allows to iterate over data (in
this context, an array of byte characters) and terminate early when
a condition is met (hence "fail-first"), and shorten the specified
vector length based on this early termination. This feature is very useful
for string search and copy functions. One great example is the \texttt{strncpy}
function, code \href{https://git.libre-soc.org/?p=openpower-isa.git;a=blob;f=src/openpower/decoder/isa/test_caller_svp64_ldst.py;h=4ecf534777a5e8a0178b29dbcd69a1a5e2dd14d6;hb=d9544764b1710f3807a9c0685d150a665f70b9a2#l31}{here}
\footnote{LibreSOC openpower commit \texttt{\#d9544764...}
(\texttt{fix vertical-first sv.bc}) is used to refer to
SVP64 spec and code examples. Commit link
\href{https://git.libre-soc.org/?p=openpower-isa.git;a=commit;h=d9544764b1710f3807a9c0685d150a665f70b9a2}{here}.}

\subsubsection{Simple-V assistance to VectorCamp and Vantosh}


\begin{enumerate}
  \item Provide assistance to VectorCamp and Vantosh with currently implemented
features of Simple-V.
  \item To find causes and/or workarounds with code bugs due to the ISACaller
simulator (the simulator which supports PowerISA 3.0 and implements the current
Simple-V specification).
  \begin{enumerate}
    \item If a bug related to the simulator is found, to create a test
  case and submit a bug report to LibreSOC (see the \texttt{sv.bc} branching address
  bug in \href{https://bugs.libre-soc.org/show_bug.cgi?id=1210}{bug \#1210}
  as an example).
    \item Document current workarounds required due to binutils support lagging
  behind current Simple-V (as a result of time/budget constraints in LibreSOC):
    \begin{enumerate}
      \item  Use of original assembler instruction required instead of
      extended mnemonics when using SimpleV \texttt{sv.} prefix.
      Examples of extended mnemonics are \texttt{nop}, \texttt{bdnz},
      \texttt{subi}, etc.). Instead have to use
      \texttt{ori 0,0,0}, \texttt{bc 16,0,target}, \texttt{addi Rx,Ry,-value}.
    \end{enumerate}
    \item \texttt{sv.bc} with \texttt{BO=0} (other \texttt{BO} modes weren't tested, see
  Figure 40. BO field encodings in section 2.4 of PowerISA 3.0 Book I
  for the full list), has to be given a manually computed address.
  See an example of how to this LINK TO "debugging function test"??
  \end{enumerate}
\end{enumerate}

\subsubsection{Standard C function SimpleV implementations}

To write implementations of several \texttt{glibc} string functions utilising a subset
of the available SimpleV feature set (due to limited binutils support).

\subsubsection{Integration into the \texttt{glibc} test environment}

VectorCamp has done the initial work of studying \texttt{glibc},
creating wrapper code, and writing up an example function, \texttt{memchr()},
for Vantosh and Red to use as a template (from which the other functions were
written up).

\subsubsection{Documentation}

Red has planned to document several critical parts of the developments for:
reproducibility; to aid understanding; and help developers to come back to
the code in future milestone and/or projects.

The documentation involves:

\begin{itemize}
  \item Environment setup
  \item Running existing test regressions
  \item Creating new function implementations
  \item Debugging function when failures occur
\end{itemize}

\section{Setting up the environment for working with glibc tests}

The setup script is available
\href{https://git.vantosh.com/ngisearch/documentation/src/branch/master/shell-scripts}{here}.
LibreSOC copy of the scripts is available
\href{https://git.libre-soc.org/?p=dev-env-setup.git;a=tree}{here}.

\subsection{Setting up the LibreSOC chroot}

Create LibreSOC chroot (host system):

\begin{verbatim}
#: cd /PATH/TO/dev-env-setup
#: ./mk-deb-chroot glibc-svp64
#: ./cp-scripts-to-chroot glibc-svp64
\end{verbatim}

Scripts inside chroot:

\begin{verbatim}
$: schroot -c glibc-svp64
(glibc-svp64)$: cd ~/dev-env-setup
(glibc-svp64)$: sudo bash
(glibc-svp64)#: ./install-hdl-apt-reqs
(glibc-svp64)#: ./binutils-gdb-install
(glibc-svp64)#: ./hdl-dev-repos
(glibc-svp64)#: exit
(glibc-svp64)$: ./ngi-search-glibc-svp64
\end{verbatim}

Each function test regression can be run by calling:

\begin{verbatim}
(glibc-svp64)$: cd ~/src/glibc-svp64/svp64-port
(glibc-svp64)$: ./SILENCELOG='!instr_in_outs' ./test-memchr-svp64 --direct
\end{verbatim}

For no simulator log, set \texttt{SILENCELOG=1}.
For full logging, skip setting \texttt{SILENCELOG}.

The full test regression may take several days (depending on your hardware).
So far, some tests fails. Some of the failures are due to a manually reduced
maximum string size used in the test (which was done for reducing the time
taken to run regressions).\\

To save the regression results and simulator log:

\begin{verbatim}
(glibc-svp64)$: ./test-memchr-svp64 --direct >& /tmp/f
\end{verbatim}

To check for failed tests, run:

\begin{verbatim}
(glibc-svp64)$: grep "Wrong" /tmp/f
\end{verbatim}

\subsection{Generate this document}

This document is written in \LaTeX{} and a PDF can be generated using provided
shell script. The script works with the LibreSOC environment, however the
actual \LaTeX{} document can be generated on any system.

\begin{verbatim}
(glibc-svp64)$: cd ~/dev-env-setup
(glibc-svp64)$: ./ngi-search-docs
\end{verbatim}

\subsubsection{Running existing test regressions}

\subsubsection{Creating new function implementations}

\subsubsection{Debugging function when failures occur}

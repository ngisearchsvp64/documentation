\section{Red Semiconductor Report Overview}

\subsection{Red's goals}

\subsubsection{Improving Simple-V spec}

\begin{enumerate}
  \item To check and improve LibreSOC's current Simple-V spec by finding/implementing
  new features relevant to processing strings and search algorithms in general
  (as part of NGI Search grant goals).
\end{enumerate}

An example feature which came out of this research is
Data-Dependent Fail-First, which allows to iterate over data (in
this context, an array of byte characters) and terminate early when
a condition is met (hence "fail-first"), and shorten the specified
vector length based on this early termination. This feature is very useful
for string search and copy functions. One great example is the \texttt{strncpy}
function, code \href{https://git.libre-soc.org/?p=openpower-isa.git;a=blob;f=src/openpower/decoder/isa/test_caller_svp64_ldst.py;h=4ecf534777a5e8a0178b29dbcd69a1a5e2dd14d6;hb=d9544764b1710f3807a9c0685d150a665f70b9a2#l31}{here}
\footnote{LibreSOC \texttt{openpower-isa} repo commit \texttt{\#d9544764...}
(\texttt{fix vertical-first sv.bc}) is used to refer to
SVP64 spec and code examples. Commit link
\href{https://git.libre-soc.org/?p=openpower-isa.git;a=commit;h=d9544764b1710f3807a9c0685d150a665f70b9a2}{here}.}

\subsubsection{Simple-V assistance to VectorCamp and Vantosh}


\begin{enumerate}
  \item Provide assistance to VectorCamp and Vantosh with currently implemented
features of Simple-V.
  \item To find causes and/or workarounds with code bugs due to the ISACaller
simulator (the simulator which supports PowerISA 3.0 and implements the current
Simple-V specification).
  \begin{enumerate}
    \item If a bug related to the simulator is found, to create a test
  case and submit a bug report to LibreSOC (see the \texttt{sv.bc} branching address
  bug in \href{https://bugs.libre-soc.org/show_bug.cgi?id=1210}{bug \#1210}
  as an example).
    \item Document current workarounds required due to binutils support lagging
  behind current Simple-V (as a result of time/budget constraints in LibreSOC):
    \begin{enumerate}
      \item  Use of original assembler instruction required instead of
      extended mnemonics when using Simple-V \texttt{sv.} prefix.
      Examples of extended mnemonics are \texttt{nop}, \texttt{bdnz},
      \texttt{subi}, etc.). Instead have to use
      \texttt{ori 0,0,0}, \texttt{bc 16,0,target}, \texttt{addi Rx,Ry,-value}.
    \end{enumerate}
    \item \texttt{sv.bc} with \texttt{BO=0} (other \texttt{BO} modes weren't tested, see
  Figure 40. BO field encodings in section 2.4 of PowerISA 3.0 Book I
  for the full list), has to be given a manually computed address.
  See an example of how to do this in Section \ref{subsec:determine_cause_err}
  on page \pageref{subsec:determine_cause_err}.
  \end{enumerate}
\end{enumerate}

\subsubsection{Standard C function Simple-V implementations}

To write implementations of several \texttt{glibc} string functions utilising a subset
of the available Simple-V feature set (due to limited binutils support).

\subsubsection{Integration into the \texttt{glibc} test environment}

VectorCamp has done the initial work of studying \texttt{glibc},
creating wrapper code, and writing up an example function, \texttt{memchr()},
for Vantosh and Red to use as a template (from which the other functions were
written up).

\subsubsection{Documentation}

Red has planned to document several critical parts of the developments for:
reproducibility; to aid understanding; and help developers to come back to
the code in future milestone and/or projects.\\

The documentation includes:

\begin{itemize}
  \item Environment setup
  \item Running existing test regressions and debugging
  \item Creating new function implementations
  \item Document discovered bugs in instruction simulator
\end{itemize}

\section{Setting up the environment for working with glibc tests}

The setup script is available
\href{https://git.vantosh.com/ngisearch/documentation/src/branch/master/shell-scripts}{here}.
LibreSOC copy of the scripts is available
\href{https://git.libre-soc.org/?p=dev-env-setup.git;a=tree}{here}.

\subsection{Setting up the LibreSOC chroot}

Create LibreSOC chroot (host system):

\begin{verbatim}
#: cd /PATH/TO/dev-env-setup
#: ./mk-deb-chroot glibc-svp64
#: ./cp-scripts-to-chroot glibc-svp64
\end{verbatim}

Scripts inside chroot:

\begin{verbatim}
$: schroot -c glibc-svp64
(glibc-svp64)$: cd ~/dev-env-setup
(glibc-svp64)$: sudo bash
(glibc-svp64)#: ./install-hdl-apt-reqs
(glibc-svp64)#: ./binutils-gdb-install
(glibc-svp64)#: ./hdl-dev-repos
(glibc-svp64)#: exit
(glibc-svp64)$: ./ngi-search-glibc-svp64
\end{verbatim}

Each function test regression can be run by calling:

\begin{verbatim}
(glibc-svp64)$: cd ~/src/glibc-svp64/svp64-port
(glibc-svp64)$: ./SILENCELOG='!instr_in_outs' ./test-memchr-svp64 --direct
\end{verbatim}

For no simulator log, set \texttt{SILENCELOG=1}.
For full logging, skip setting \texttt{SILENCELOG}.

The full test regression may take several days (depending on your hardware).
So far, some tests fails. Some of the failures are due to a manually reduced
maximum string size used in the test (which was done for reducing the time
taken to run regressions).\\

To save the regression results and simulator log:

\begin{verbatim}
(glibc-svp64)$: ./test-memchr-svp64 --direct >& /tmp/f
\end{verbatim}

To check for failed tests, run:

\begin{verbatim}
(glibc-svp64)$: grep "Wrong" /tmp/f
\end{verbatim}

\subsection{Generate this document}

This document is written in \LaTeX{} and a PDF can be generated using provided
shell script. The script works with the LibreSOC environment, however the
actual \LaTeX{} document can be generated on any system.

\begin{verbatim}
(glibc-svp64)$: cd ~/dev-env-setup
(glibc-svp64)$: ./ngi-search-docs
\end{verbatim}

% SPDX-License-Identifier: LGPL-3-or-later
% Copyright 2023 Red Semiconductor Ltd.
%
% Funded by NGI Search Programme HORIZON-CL4-2021-HUMAN-01 2022,
% https://www.ngisearch.eu/, EU Programme 101069364.

\chapter{Debugging and Running Function Tests}
\label{sec:debug_run_tests}

\section{Running tests and checking if errors present}

Figuring out the issue with a function is an iterative process, and may require
looking at many different files.\\

For this guide, \texttt{strchr} will be used (as it produced errors at the time).\\

To start with, run compile and run the test cases:

\begin{verbatim}
(glibc-svp64)$: cd ~/src/glibc-svp64/svp64-port/
(glibc-svp64)$: make clean
(glibc-svp64)$: make all
(glibc-svp64)$: SILENCELOG='!instr_in_outs' ./test-strchr-svp64 --direct \
                >& /tmp/f
\end{verbatim}

(The \texttt{'!instr\_in\_outs'} setting for the \texttt{SILENCELOG}
environment variable reduces the logging coming out of ISACaller,
which usually isn't required for day-to-day testing. Full logging
however is useful when trying to determine if the instruction
behaviour is correct.)\\

The test results can be found in the temporary file \texttt{f} under
\texttt{/tmp/}. The file can be opened and inspected while the test regression
is running (although it's best to use a lightweight text editor like
\texttt{vim} as the file will grow quickly).\\

To check if there were any failures without opening the test result file:

\begin{verbatim}
(glibc-svp64)$: echo $?
\end{verbatim}

The value of the \texttt{?} variable will be non-zero if there are errors. Test
regression failures are caused by a mismatch between the reference C
implementation of \texttt{strchr} and the Simple-V implementation.\\

Inside the result tests file for the occurance of
\texttt{Wrong result in function}.
The version of \texttt{strchr\_svp64} used for this example is in git commit
\texttt{\#9378006a}. Simple-V implementation
\href{https://github.com/ngisearchsvp64/glibc-svp64/blob/9378006a84bdef6af85eb0f810fb62fedc62c588/svp64-port/svp64/strchr_svp64.s}{strchr\_svp64.s}.
Three errors were found in the first group of tests,
\begin{verbatim}
do_test (0, 16 << i, MAX_SIZE, SMALL_CHAR, MIDDLE_CHAR);
\end{verbatim}

See,
\href{https://github.com/ngisearchsvp64/glibc-svp64/blob/9378006a84bdef6af85eb0f810fb62fedc62c588/svp64-port/test-strchr.c#L267}.

\section{Determining causes of errors}
\label{subsec:determine_cause_err}

\subsubsection{Incorrect value for character \textbf{c} loaded into \texttt{GPR[4]}}

Looking at the intial debug printout, one can see that the give byte/character
\textbf{c} is supposed to be \texttt{0xfe}, however the value being printed
is \texttt{ffffffff}, while the value being stored in register \texttt{GPR[4]}
is \texttt{0xfffffffffffffffe}.

\begin{verbatim}
strchr called: s: 0x7ffd93fd8ad0, c: fe
strchr_svp64 called: s: 0x7ffd93fd8ad0, c: fe
homeIsaDir: /home/am/src/openpower-isa/src/openpower/decoder/isa/
bytes: 0, bytes_rem: 1
ffffffff
binary <_io.BytesIO object at 0x7f74e7e3f0a0>

Memory:
0x00100000:  FF 00 00 00 00 00 00 00  00 00 00 00 00 00 00 00

call ori ori
read reg 4/0: 0xfffffffffffffffe
read reg 8/0: 0x0
write reg r(8, 8, 0) 0xfffffffffffffffe ew 64
reg  0 00000000 00000000 00000000 00100000 fffffffffffffffe ...
reg  8 fffffffffffffffe 00000000 00000000 00000000 00000000 ...
\end{verbatim}

Both wrapper and test C files were fixed to pass in a single byte, see
\href{https://github.com/ngisearchsvp64/glibc-svp64/commit/8f1b25340ee2f108027a6f50e365d42aeb7cc939}{commit}.

\subsubsection{Wrong return value}

The failed test in question is the second one in the test regression. Can be
found by searching for the second occurance of the term \texttt{Memory}
(shows the full string to be tested loaded in memory, line \texttt{\#48602}),
and by searching for the first occurance of \texttt{Wrong} term
(the end of the test, line \texttt{\#48510}).\\

(Given line numbers based on \texttt{SILENCELOG='!instr\_in\_outs'} value.)\\

Start of test:

\begin{verbatim}
strchr_svp64 called: s: 0x7ff622cbd000, c: 17(^W)
bytes: 256, bytes_rem: 0
\end{verbatim}

String loaded into memory (only first 48 bytes shown):

\begin{verbatim}
    Memory:
    0x00100000:  20 37 4E 65 7C 93 2B 42  59 70 87 9E 36 4D 64 7B
    0x00100010:  92 2A 41 58 6F 86 9D 35  4C 63 7A 91 29 40 57 6E
    0x00100020:  17 9C 34 4B 62 79 90 28  3F 56 6D 84 9B 33 4A 61
    0x00100030:  ...
\end{verbatim}

End of test:

\begin{verbatim}
return val        : 0000000000000000
./test-strchr-svp64: Wrong result in function
                     strchr_svp64 0x17 (nil) 0x7ff622cbd020
\end{verbatim}

The return value should be \texttt{100020}, because that's where
the char \texttt{0x17} occurs.\\

After looking through the instruction listings and register values in the
results file, it was discovered that the \texttt{sv.bc} branch address
was incorrect.\\

Place of interest starts at line \texttt{\#26887}, most empty register
lines ommitted).

\begin{verbatim}
call mtspr mtspr
read reg 6/0: 0x5
reg  0 00000000 00000000 00000000 00100020 00000017 000000e0 00000005 00000004
reg  8 1717171717171717 0000 0000 0000 00100020 00100028 00100030 00100038
reg 16 289079624b349c17 614a339b846d563f 9a836c553e278f78 543d268e77604932 0
... skip reg values ...
call bc bc
write reg CTR 0x4
reg  0 00000000 00000000 00000000 00100020 00000017 000000e0 00000005 00000004
reg  8 1717171717171717 0000 0000 0000 00100020 00100028 00100030 00100038
reg 16 289079624b349c17 614a339b846d563f 9a836c553e278f78 543d268e77604932 0
... skip reg values ...
call subf subf
read reg 7/0: 0x4
read reg 6/0: 0x5
read reg 6/0: 0x5
write reg r(6, 6, 0) 0x1 ew 64
reg  0 00000000 00000000 00000000 00100020 00000017 000000e0 00000001 00000004
reg  8 1717171717171717 0000 0000 0000 00100020 00100028 00100030 00100038
reg 16 289079624b349c17 614a339b846d563f 9a836c553e278f78 543d268e77604932 0
... skip reg values ...
\end{verbatim}

The \texttt{mtspr} instruction corresponds to \texttt{mfspr}
(an extended mnemonic) listing,
\href{https://github.com/ngisearchsvp64/glibc-svp64/blob/9378006a84bdef6af85eb0f810fb62fedc62c588/svp64-port/svp64/strchr_svp64.s#L108}{line \#108},
of the assembly code.\\

The correct behaviour is on a successful byte match (in this case
matches \texttt{0x17} as expected) there should be a branch to line
\texttt{\#108}, however the simulator jumps to the \texttt{subf}
(assembly using \texttt{sub} extended mnemonic),
\href{https://github.com/ngisearchsvp64/glibc-svp64/blob/9378006a84bdef6af85eb0f810fb62fedc62c588/svp64-port/svp64/strchr_svp64.s#L112}{line \#112}.\\

\label{subsubsec:wrong_val_man_addr}

Turns out the manual \texttt{sv.bc} branch address wasn't update after
copying assembler from existing function.\\

Due to binutils not fully supporting \texttt{sv.bc}, an incorrect
branch address is generated when a label is given in the assembly file.\\

For now, the branch address must be calculated manually.
The method for doing this:

\begin{itemize}
  \item Run the makefile to generate \texttt{strchr\_svp64.o}.
  \item Generate an assembly listing.
\end{itemize}

Command to do so:

\begin{verbatim}
(glibc-svp64)$: powerpc64le-linux-gnu-objdump -D svp64/strchr_svp64.o \
                > svp64/dump_strchr.txt
\end{verbatim}


\begin{itemize}
  \item Open the file and look at where the \texttt{sv.bc} instruction starts.
\end{itemize}

The dump section in question:

\begin{verbatim}
0000000000000054 <.inner>:
  54:   78 1b 6c 7c     mr      r12,r3
  ... skip instructions ...
  84:   00 20 40 05     .long 0x5402000
  88:   1c 00 02 40     bdnzf   eq,a4 <.determine_loc+0xc>
  ... skip instructions ...

0000000000000098 <.determine_loc>:
  98:   a6 02 e9 7c     mfctr   r7
  9c:   04 00 c0 38     li      r6,4
  a0:   50 30 c7 7c     subf    r6,r7,r6
  a4:   00 00 26 2c     cmpdi   r6,0
  ... skip instructions ...
\end{verbatim}

The \texttt{sv.bc} instruction starts at address \texttt{0x84}, because
the prefix is included as part of the instruction for the calculation of the
relative branch address.\\

The branch address shown by the object dump (\texttt{0xa4}) is 4 bytes higher
than the actual branch address taken (during simulation, the branch jumps to
where the \texttt{subf} instruction occurs, address \texttt{0xa0}.


\begin{itemize}
  \item Find the desired address to jump to. Looking at the object dump,
  this will correspond to address \texttt{0x98}, or where the
\texttt{.determine\_loc} label is.
  \item Calculate the offset to use by subtracting the start of \texttt{sv.bc}
  from the desired address.
\end{itemize}

Calculation:

\begin{verbatim}
addr_target - addr_sv.bc = 0x98 - 0x84 = 0x14
\end{verbatim}

Thus the \texttt{target\_address} parameter in \texttt{sv.bc}
(last argument) needs to be changed to {0x14}.

The code has been fixed in
\href{https://github.com/ngisearchsvp64/glibc-svp64/commit/2d2c0f70dc5cca10a1c5d92d726406903f9e5b23}.


\section{Guide to New SVP64 Implementation of a GLIBC Function}
\label{sec:adding_new_func}

The directory of the glibc repo is the same as for initial environment setup:

\begin{verbatim}
(glibc-svp64)$: cd ~/src/glibc-svp64/glibc
\end{verbatim}

\subsection{Required steps - \texttt{strchr} as an example}

\subsubsection{Copy \texttt{test-[function].c} from glibc}

From glibc, copy the test C file for the respective function
into \texttt{svp64-port} directory.

It should be under the function category, for example, strchr is
under \texttt{string/test-strchr.c}:

\begin{verbatim}
(glibc-svp64)$: cp ~/src/glibc-svp64/glibc/string/test-strchr.c \
                ~/src/glibc-svp64/svp64-port/
\end{verbatim}

\subsubsection{Making adjustments to \texttt{test-[function].c}}

Add the following \texttt{\#define}'s after the
\texttt{typedef CHAR *(*proto\_t) (const CHAR *, int);} line:

\begin{verbatim}
#define STRCHR_SVP64 strchr_svp64
#define MAX_SIZE    256
CHAR *SIMPLE_STRCHR (const CHAR *, int, size_t);
CHAR *STRCHR_SVP64 (const CHAR *, int, size_t);
\end{verbatim}

\texttt{MAX\_SIZE} is set to 256 to reduce the time taken to run
the regression tests.\\

The \texttt{IMPL} calls will need to be modified as they define the
functions used during the test. There was an inconsistency with the
upper/lower case \texttt{simple\_} prefix, but that might just be a
historical reason.\\

Comment out:

\begin{verbatim}
IMPL (stupid_STRCHR, 0)
IMPL (simple_STRCHR, 0)
IMPL (STRCHR, 1)
\end{verbatim}

And add:

\begin{verbatim}
IMPL (STRCHR_SVP64, 1)
IMPL (simple_STRCHR, 2)
\end{verbatim}

For debugging purposes, add a printf statement as first line inside
\texttt{SIMPLE\_[FUNCTION]} function code.
For \texttt{strchr} the statement is added inside \texttt{SIMPLE\_STRCHR}:

\begin{verbatim}
printf("strchr called: s: %p, c: %02x(%c)\n", s, (uint8_t)c, c);
\end{verbatim}

For initial testing, it's worthwhile to disable most tests,
and only turn on a few. The C function \texttt{test\_main} at the bottom
of the file contains the tests being run.\\

\texttt{do\_test} is the function used for running a single test case, and
has five arguments (for the \texttt{strchr)}:

\begin{itemize}
  \item \texttt{align} - byte alignment in the array
  \item \texttt{pos} - position in the array
  \item \texttt{len} - length of char array
  \item \texttt{seek\_char} - when \texttt{align} and \texttt{pos} set to 0,
  this is equal to the character the function will search for.
  \item \texttt{max\_char} - Largest permitted character (buffer char is
  limited by performing modulo \texttt{max\_char}).
\end{itemize}

In the test cases done for \texttt{memchr}, \texttt{memrchr}, \texttt{strchr},
the length argument is limited to 256 for reducing the time take to run tests.\\

Inside the \texttt{SIMPLE\_[FUNCTION]} function, add the printf statement
for debug and/or logging:

\begin{verbatim}
printf("strchr called: s: %p, c: %02x(%c)\n", s, (uint8_t)c, c);
\end{verbatim}

\subsubsection{\texttt{[function]\_wrapper.c}}

Create a new \texttt{strchr\_wrapper.c} C file which will interface with
the glibc tests and access the ISACaller PowerISA+SVP64 simulator. An existing
\texttt{[function]\_wrapper.c} file can be used (in this example,
\texttt{memchr}):

\begin{verbatim}
(glibc-svp64)$: cp ~/src/glibc-svp64/svp64-port/memchr_wrapper.c \\
                ~/src/glibc-svp64/svp64-port/strchr_wrapper.c
\end{verbatim}

\subsubsection{Making adjustments to \texttt{[function]\_wrapper.c}}

Sadly this is difficult to automate (at least for now), because other than
substituting the function name, the logic of the code may need to change.
The general structure will however remain the same.

\begin{itemize}
  \item The easiest change to make is to replace every instance of previous
  function name to the new one being implemented. With the current example,
  replace \texttt{memchr} with \texttt{strchr},
  and \texttt{MEMCHR} with \texttt{STRCHR}.
  \item Change the input arguments of \texttt{[FUNCTION]\_SVP64}.
  In this case, only \texttt{s} and \texttt{c} are necessary
  (as string function continues until a null byte is encountered.
  \item Update the code and comments of the \texttt{[FUNCTION]\_SVP64}.
  In this case, remove use of \texttt{n}. \texttt{size\_t bytes} need to be
  calculated using \texttt{strlen(s)} because the length of the string
  is not provided.
  \item Make sure to update the copyright notice at the top.
\end{itemize}

\subsubsection{Adjusting Makefile}

The Makefile in \texttt{svp64-port} needs to be updated to include the new
function test code (in this example, \texttt{strchr}).

Add a new target:

\begin{verbatim}
strchr_TARGET	= test-strchr-svp64
\end{verbatim}

Below \texttt{BINDIR} variable add:
\begin{verbatim}
strchr_CFILES	:= support_test_main.c test-strchr.c strchr_wrapper.c
strchr_ASFILES := $(SVP64)/strchr_svp64.s $(SVP64)/strchr_orig_ppc64.s
strchr_SVP64OBJECTS := $(strchr_ASFILES:$(SVP64)/%.s=$(SVP64)/%.o)
strchr_OBJECTS := $(strchr_CFILES:%.c=%.o)
strchr_BINFILES := $(BINDIR)/strchr_svp64.bin
strchr_ELFFILES := $(BINDIR)/strchr_svp64.elf
\end{verbatim}

Add target for the \texttt{test-[function].o} object:
\begin{verbatim}
test-strchr.o: test-strchr.c
	$(CC) -c $(CFLAGS) -DMODULE_NAME=testsuite -o $@ $^
\end{verbatim}

Add target for generating assembly implementation of using standard PowerISA:
\begin{verbatim}
$(SVP64)/strchr_orig_ppc64.s: $(GLIBCDIR)/string/strchr.c
	$(CROSSCC) $(CROSSCFLAGS) -S -g0 -Os -DMODULE_NAME=libc -o $@ $^
\end{verbatim}

Append \texttt{\$(strchr\_TARGET)} to the \texttt{all} make rule.

Add a target for the final \texttt{test-[function]-svp64} binary:
\begin{verbatim}
$(strchr_TARGET): $(strchr_OBJECTS) $(strchr_SVP64OBJECTS) $(strchr_ELFFILES) \
$(strchr_BINFILES)
	$(CC) -o $@ $(strchr_OBJECTS) $(LDFLAGS)
\end{verbatim}

Add a line to the \texttt{clean} make rule:
\begin{verbatim}
$ rm -f $(strchr_OBJECTS) $(strchr_SVP64OBJECTS) $(strchr_BINFILES) \
$(strchr_ELFFILES) $(strchr_TARGET)
\end{verbatim}

Append \texttt{\$(strchr\_TARGET)} to the line under the \texttt{remove} make rule.

\subsubsection{\texttt{[function]\_svp64.s} assembler file}

This file can be started by copying from existing SVP64 function, or by using
the generated assembler using the reference implementation
(\texttt{[function]\_orig\_ppc64.s}), although the generated assembler is
probably more difficult to follow than simply writing from scratch.\\

If copying from the \texttt{memchr} SVP64 assembler:
\begin{verbatim}
(glibc-svp64)$: cp ~/src/glibc-svp64/svp64-port/svp64/memchr_svp64.s \\
                ~/src/glibc-svp64/svp64-port/svp64/strchr_svp64.s
\end{verbatim}

Modifications needed to be made:
\begin{itemize}
  \item Change \texttt{memchr} string to \texttt{strchr}.
  \item Make modifications based on the operation of the new function
  (and difference in input arguments).
\end{itemize}

% SPDX-License-Identifier: LGPL-3-or-later
% Copyright 2023 Red Semiconductor Ltd.
%
% Funded by NGI Search Programme HORIZON-CL4-2021-HUMAN-01 2022,
% https://www.ngisearch.eu/, EU Programme 101069364.

\chapter{Issues Discovered During Development}

\section{Incorrect Next Instruction Address (NIA) in `sv.bc` branch instruction}

\begin{itemize}
  \item Raised bug on the Libre-SOC bug tracker:
  \href{https://bugs.libre-soc.org/show_bug.cgi?id=1210}{bug \#1210}
  \item Libre-SOC commit \#d9544764 which fixed this issue:
  \href{https://git.libre-soc.org/?p=openpower-isa.git;a=commitdiff;h=d9544764b1710f3807a9c0685d150a665f70b9a2}{here}
  \item Libre-SOC unit test which demonstrated this issue:
  \href{https://git.libre-soc.org/?p=openpower-isa.git;a=blob;f=src/openpower/decoder/isa/test_caller_svp64_bc.py;h=93689ded619f8fa67b455f18b122fa60220ddea1;hb=089e6d352ec57be4ab645d18ad9e95df3af0d365#l310}{here}
\end{itemize}

The \acrshort{SVP64} implementation of \texttt{memchr} uses the vectorised branch
conditional instruction, \texttt{sv.bc} in the \acrshort{SVP64} Vertical-First
\href{https://github.com/ngisearchsvp64/glibc-svp64/blob/1afb94889b8ea2f85844e410f87e5a9b8e2e959f/svp64-port/svp64/memchr_svp64.s#L67}{loop}.\\

The instruction below is the Simple-V form of the normal \texttt{bc} instruction.
It works by checking a single bit in the \acrfull{CR}
(see Section 2.3.1 of Book I,
\href{https://openpower.foundation/specifications/isa/}{PowerISA v3.0C spec})
for more information on \acrshort{CR}).
The branch is taken depending on the set \texttt{BO} mode (which may also
involve the \acrfull{CTR} (Section 2.3.3) and the CR bit specified.

\begin{verbatim}
sv.bc               0, *2, .found
\end{verbatim}

Has a \texttt{BO=0} mode which means "Decrement the CTR, then branch
if the decremented CTRM:63 $\neq$ 0 and CRBI=0". (See Figure 40, Section 2.4
Branch Instructions for the full list of \texttt{BO} modes.).\\

\texttt{CR} is split into 8 4-bit CR Fields 0-7, where every 4-bits correspond
to (copied from Section 2.3.1 of
\href{https://openpower.foundation/specifications/isa/}{PowerISA spec}):

\begin{itemize}
  \item 0: Negative (LT), The result is negative.
  \item 1: Positive (GT), The result is positive.
  \item 2: Zero (EQ), The result is zero.
  \item 3: Summary Overflow (SO)
\end{itemize}

So this instruction is supposed to branch only \textit{if}:

\begin{itemize}
  \item CTR value after decrementing is greater than zero
  \item EQ bit of CR Field 0 (bit 2+32 of CR) is zero.
At every consecutive iteration of the Simple-V loop (in this case,
the Simple-V index is going up from 0 to (VL-1)), the CR Field will move up
by 1 (i.e. CR Field 0, CR bit 2+32; CR Field 1, CR bit 6+32;
CR Field 2, CR bit 10+32; CR Field 3, and so on).
\end{itemize}

In the context of \texttt{memchr}, the branch must occur if one of the 8 bytes
in registers \texttt{s} (starts at \texttt{GPR[16]} increases up to
\texttt{`GPR[19]}) contains the matching byte \texttt{c}
(in \texttt{GPR[4]}).\\

Before Libre-SOC's ISACaller simulator was fixed, the following would occur:
the \texttt{CR} bit will be set correctly by the compare immediate instruction
\texttt{sv.cmpi}, but during the \texttt{sv.bc} instruction, the simulator
will proceed to the next instruction address (\texttt{svstep})
instead of branching to \texttt{.found}.

\subsubsection{Replicating this issue}

It is assumed the chroot environment has already been setup.

(Using the commit \texttt{\#089e6d35} of the \texttt{openpower-isa} repo:
\href{https://git.libre-soc.org/?p=openpower-isa.git;a=commitdiff;h=089e6d352ec57be4ab645d18ad9e95df3af0d365}{here}).

\begin{verbatim}
(glibc-svp64)$: cd ~/src/openpower-isa
(glibc-svp64)$: git checkout 089e6d352ec57be4ab645d18ad9e95df3af0d365
(glibc-svp64)$: make
\end{verbatim}

Running \texttt{make} will regenerate the necessary files for the simulator.

To run the standalone unit test written to demonstrate the issue:
\begin{verbatim}
(glibc-svp64)$: cd ~/src/openpower-isa/src/openpower/decoder/isa
(glibc-svp64)$: python3 test_caller_svp64_bc.py >& /tmp/f
\end{verbatim}

The test is called \texttt{test\_sv\_branch\_vertical\_first()}. Other tests inside
\texttt{test\_caller\_svp64\_bc.py} could be disabled by changing the
\texttt{test\_} prefix to something else, and they will be ignored.

In the results file, the following message will be present:
\begin{verbatim}
FAIL: test_sv_branch_vertical_first (__main__.DecoderTestCase)
this is a branch-vertical-first-loop demo which shows an early
----------------------------------------------------------------------
Traceback (most recent call last):
  File "test_caller_svp64_bc.py", line 341, in test_sv_branch_vertical_first
    self.assertEqual(sim.spr('CTR'), SelectableInt(1, 64))
AssertionError: SelectableInt(value=0x0, bits=64) !=
                SelectableInt(value=0x1, bits=64)
\end{verbatim}

And the reason for the incorrect \acrshort{CTR} value, is because instead of the expected
branch, another iteration of the \acrfull{SV} loop occurs, thus
decrementing \acrshort{CTR} to 0.

To run the \texttt{memchr} tests:
\begin{verbatim}
(glibc-svp64)$: cd ~/src/glibc-svp64/svp64-port/
(glibc-svp64)$: make clean
(glibc-svp64)$: make all
(glibc-svp64)$: SILENCELOG='!instr_in_outs' ./test-memchr-svp64 --direct \
                >& /tmp/f
\end{verbatim}


Searching for the first occurance of \texttt{Wrong}, the following is found on
\texttt{line \#31233}:
\begin{verbatim}
return val        : 0000000000000000
./test-memchr-svp64: Wrong result in function memchr_svp64 (nil) 0x7f2abc182020
\end{verbatim}

The returned value from the \acrshort{SVP64} implementation was 0, while the reference C
gave 0x7f2abc182020 (the correct result).\\

Jump to the occurance of \texttt{Memory:} which occurred before the error
message (\texttt{line \#461}). The memory printout (only first 48 bytes shown):
\begin{verbatim}
Memory:
0x00100000:  01 18 2F 46 5D 74 0C 23  3A 51 68 7F 18 2E 45 5C
0x00100010:  73 0B 22 39 50 67 7E 16  2D 44 5B 72 0A 21 38 4F
0x00100020:  17 7D 15 2C 43 5A 71 09  20 37 4E 65 7C 14 2B 42
0x00100030:  ...
\end{verbatim}

(The simulator's memory address space is different from the native C,
but the \texttt{memchr\_wrapper.c} code takes care of the conversion.)\\

To find where the incorrect branch occurred, search for
\texttt{9715a432c157d17} (in reverse order compared to memory print because of
Big/Little-Endian conversion). The first occurance of this value is when
the doubleword load \texttt{ld} is called. This corresponds to
\href{https://github.com/ngisearchsvp64/glibc-svp64/blob/1afb94889b8ea2f85844e410f87e5a9b8e2e959f/svp64-port/svp64/memchr_svp64.s#L64}{line \#64}
of the \texttt{memchr} SVP64 assembler.\\

The instructions called after \texttt{ld} are:

\begin{itemize}
  \item Compare bytes \texttt{sv.cmpb} (generates the mask \texttt{0xff}
        because \texttt{0x17} is the character to match for)
  \item Compare immediate \texttt{sv.cmpi} (sets bit 1 (GT), and clears
        bit 2 (EQ), of \acrshort{CR} Field 0)
  \item Branch conditional (decrements \acrshort{CTR}, tests for bit2=0, but
        doesn't branch)
  \item \acrshort{SV} step \texttt{svstep} (branch was not taken,
        \textit{this shouldn't have happened})
\end{itemize}

\subsubsection{Debugging with Python debugger}

Add the following lines to \texttt{caller.py} in the
`~/src/openpower-isa/src/openpower/decoder/isa/` directory.
\begin{verbatim}
import pdb; pdb.set_trace() # Add at line #15
breakpoint()                # Add at line #2361 inside the call() method
\end{verbatim}

The breakpoint in \texttt{caller.py} is not necessary at first, as it will
stop every time an instruction is called (it will take a while until the
above scenario occurs).\\

The main place where breakpoint should be set is in:

\begin{verbatim}
~/src/openpower-isa/src/openpower/decoder/isa/generated/svbranch.py
\end{verbatim}

after \texttt{line \#18}. This breakpoint will stop at
every call of \texttt{sv.bc}.\\

To run:
\begin{verbatim}
(glibc-svp64)$: cd ~/src/glibc-svp64/svp64-port/
(glibc-svp64)$: SILENCELOG='!instr_in_outs' ./test-memchr-svp64 --direct
\end{verbatim}

The Python intepreter will stop at the very beginning of \texttt{caller.py}.
Enter \texttt{c} (continue). The intepreter will stop at the first instance
of \texttt{sv.bc}. Continue until the memory address in register
\texttt{GPR[3]} is equal to \texttt{00100020}.
After entering \texttt{c} another 4 times, the intepreter will stop at the
call where the branch doesn't occur.\\

Now the python code can be single-stepped using \texttt{n} (next) and
\texttt{s} (step, steps into functions instead of just running
and returning result).\\

The pseudo-code for \texttt{sv.bc} can be stepped through,
confirming that all the necessary conditions for the branch occur.
Compare with the pseudo-code found in
`~/src/openpower-isa/openpower/isa/svbranch.mdwn`,
\href{https://git.libre-soc.org/?p=openpower-isa.git;a=blob;f=openpower/isa/svbranch.mdwn;h=e8b46e7700b44c6112ee2d873cc2e04b3c732370;hb=089e6d352ec57be4ab645d18ad9e95df3af0d365}{repo},
also mirrored on the
\href{https://libre-soc.org/openpower/isa/svbranch/}{Libre-SOC wiki}.\\

The generated pseudo-code function correctly updates the \acrfull{NIA},
but it is later overwritten by the call to 
\begin{verbatim}
yield from self.do_nia(asmop, ins_name, rc_en, ffirst_hit)
\end{verbatim}

On \texttt{line \#2362} of \texttt{caller.py}. For more details of the fix,
please see the Libre-SOC
\href{https://bugs.libre-soc.org/show_bug.cgi?id=1210}{bug 1210}.


\section{\texttt{memchr} Function Breakdown}

\subsection{What is \texttt{memchr}?}

\begin{verbatim}
void *memchr(const void *s, int c, size_t n);
\end{verbatim}

The function \texttt{memchr} is part of the standard C library, and can be used
after adding an \texttt{\#include <string.h>}.

\begin{verbatim}
The memchr function locates the first occurrence of c
(converted to an unsigned char) in the initial n characters
(each interpreted as unsigned char) of the object pointed to by s.
The implementation shall behave as if it reads the characters sequentially
and stops as soon as a matching character is found.

The memchr function returns a pointer to the located character,
or a null pointer if the character does not occur in the object.
\end{verbatim}

To summarise: find the first occurance of unsigned char \texttt{c} in the
array of chars pointed to by \texttt{s}. No match returns a null.

\subsection{\texttt{memchr\_wrapper.c} - C wrapper to interface glibc test with SVP64 code}

Source code for \texttt{memchr\_wrapper.c} can be found in the
\texttt{glibc-svp64} repo, path is:
\begin{verbatim}
svp64-port/memchr_wrapper.c
\end{verbatim}

This wrapper file defines a wrapper function \texttt{MEMCHR\_SVP64()} with the
same input arguments as standard \texttt{memchr()}, but also includes
register setup and call to \texttt{pypowersim} which will run
LibreSOC's \texttt{ISACaller} simulator only for
the SVP64 \texttt{memchr} implementation.\\

The string to be tested is copied from the native C address space, to the
simulator's memory area (this is why during the execution of \texttt{memchr},
a different string address will be seen, although the offsets within the
string are the same).

\subsection{\texttt{test-memchr.c} - Test regressions for \texttt{memchr}}

Source code for \texttt{test-memchr.c} can be found in the
\texttt{glibc-svp64} repo, path is:
\begin{verbatim}
svp64-port/test-memchr.c
\end{verbatim}

This file is copied from glibc, and requires several modifications to support
the SVP64 implementation. See Section \ref{sec:adding_new_func},
page \pageref{sec:adding_new_func}

\subsection{\texttt{memchr\_svp64.s} - SVP64 assembler implementation of \texttt{memchr}}

Source code for \texttt{memchr\_svp64.s} can be found in the
\texttt{glibc-svp64} repo, path is:
\begin{verbatim}
svp64-port/svp64/memchr_svp64.s
\end{verbatim}

\subsubsection{Setup}

Setup variable names (set symbol) corresponding to \acrfull{GPR} which are used
during the algorithm.

\begin{itemize}
  \item Convience register labels: \texttt{in\_ptr}, \texttt{c}, \texttt{n},
  \texttt{tmp}, \texttt{ctr}, \texttt{s}, \texttt{t}.
\end{itemize}

A macro is defined which takes the character to match, \texttt{c},
and produces an 8-byte mask with 8 copies of \texttt{c}.
This allows to compare 8 bytes of the input string at a time
(when loading 8-bytes from memory).

\begin{itemize}
  \item Example: `c=0x17`, macro will produce the mask `0x1717171717171717`.
\end{itemize}

\subsubsection{Outer loop}

\begin{enumerate}
  \item Must determine if number of remaining chars, \texttt{n} of
  string is zero. If so, return null (no match found).
  \item If \texttt{n} is less than 32, proceed to the byte search part of
  the algorithm (\texttt{.found}).
  \item If \texttt{n} is 32 or greater, can use the SVP64 Vertical-First
  routine, \texttt{.inner}. If this is the case, do the preamble
  before \texttt{.inner}:
  \begin{enumerate}
    \item First store value of \acrshort{CTR} in \texttt{ctr}.
    Configure the \acrshort{CTR}. \acrshort{CTR} is used to keep
    track of loop iterations (this is standard PowerISA use,
    not specific to SVP64). \acrshort{CTR} is also used for
    certain conditional branch modes.
    \item Setup \acrshort{SVSTATE} register using \texttt{setvl}.
    \acrfull{VL} is 4 (value in \texttt{ctr}), \acrfull{VF} mode is used
    (step over instructions before
    incrementing \texttt{srcstep}, \texttt{dststep} to the next element).
    For more detailed info on \acrshort{VF},
    \href{https://libre-soc.org/openpower/sv/svstep/}{see wiki page}.
  \end{enumerate}
\end{enumerate}

\subsubsection{Inner loop}

Comprises the inner loop of the SVP64 Vertical-First algorithm.

\begin{itemize}
  \item Load doubleword (8 bytes) of the string from memory and store in 
register starting at \texttt{s} (actual register used is \texttt{16+dststep}).
  \item Compare byte by byte the loaded 8 bytes of the string with the matching
char mask. If any of the 8 bytes match, the instruction will set
the corresponding byte to \texttt{0xff} in \texttt{t}
(starts at \texttt{GPR[32]}).
\end{itemize}

Example:
\begin{verbatim}
call cmpb cmpb
read reg 16/0: 0x9715a432c157d17
read reg 8/0: 0x1717171717171717
read reg 32/0: 0x0
write reg r(32, 32, 0) 0xff ew 64
\end{verbatim}

\begin{itemize}
  \item Compare immediate value 0 with \texttt{t}. Set the \acrshort{CR} Field
  (starting at CR Field 0). If a matching char is present in the 8-byte
  segment of the string, \texttt{t} will have a non-zero value, thus the
  CR Field EQ bit will be cleared. This bit will be used to determine whether
  to branch to the end of the algorithm, or to continue.
  \item Vectorised branch conditional. In the event of a matching byte,
  EQ bit of CR Field will be zero, and thus a branch to \texttt{.found}
  will occur if CTR is also non-zero.
  \item If no matches have occurred, continue by incrementing element using
  the \texttt{svstep} instruction. More info
  \href{https://libre-soc.org/openpower/sv/svstep/}{here}.
  \item Increment address stored in \texttt{in\_ptr} by 8, as the code checked
  8 bytes of string, and now can continue on to the next portion of the string.
  Decrement \texttt{n} by 8 as there are now 8 fewer bytes to check.
  \item Branch back to the start of \texttt{.inner} if CTR is non-zero
  (haven't checked all 32 bytes yet). Otherwise branch back to \texttt{.outer}
  to determine how to continue.
\end{itemize}

\subsubsection{Found loop}

This code is reached if:

\begin{itemize}
  \item There are fewer than 32 bytes to process, or,
  \item A match has been found in \texttt{.inner} and need to determine the
  exact byte address
\end{itemize}

The algorithm is similar to the one used in \texttt{.inner}, except that code tests
one byte at a time. If there's an exact match, then the code returns from
\texttt{memchr} function call.

\subsubsection{Tail}

If this portion of code is reached (or jumped to), then no
match has been found. Null is returned.

\subsubsection{Post-analysis issues discovered}

\begin{itemize}
  \item CTR register is decremented twice in the \texttt{inner} loop, once by
  \texttt{sv.bc}, and again by \texttt{bdnz}.
  This means the number of iterations is only 2 (as opposed to 4).
  \item \texttt{svstep} third argument needs to be set to 1 to enable
  element incrementing. Enabling this however, caused a strange issue where the
  Effective Address (EA) used in \texttt{sv.ld} was not being changed even
  after the value stored in \texttt{in\_ptr} was incremented by 8
  (EA was effectively frozen for the duration of Vertical-First).
\end{itemize}

% SPDX-License-Identifier: LGPL-3-or-later
% Copyright 2023 VectorCamp
% Copyright 2023 Red Semiconductor Ltd.
% Copyright 2023 VanTosh
%
% Funded by NGI Search Programme HORIZON-CL4-2021-HUMAN-01 2022,
% https://www.ngisearch.eu/, EU Programme 101069364.

\section{\texttt{memrchr} Function Breakdown}

\subsection{What is \texttt{memrchr}?}

\begin{verbatim}
void *memrchr(const void *s, int c, size_t n);
\end{verbatim}

This function behaves similarly to \texttt{memchr}, but searches in reverse.
In terms of the implementation, this requires a bit of additional arithmetic
to start looking in reverse.\\

As demonstration of \acrshort{SVP64}, a \textit{Horizontal-First implementation is used
instead of Vertical-First}.\\

(Sections similar to \texttt{memchr} are not covered in this section.)

\subsection{\texttt{memrchr\_svp64.s} - SVP64 assembler implementation of \texttt{memrchr}}

Source code for \texttt{memrchr\_svp64.s} can be found in the
\texttt{glibc-svp64} repo, path is:
\begin{verbatim}
svp64-port/svp64/memrchr_svp64.s
\end{verbatim}

\subsubsection{Setup}

In addition to code from \texttt{memchr}, adjust \texttt{in\_ptr}
to start at N-1 (the last char of the string).

\subsubsection{Outer loop}

\begin{enumerate}
  \item Must determine if number of remaining chars, \texttt{n} of
  string is zero. If so, return null (no match found).
  \item If \texttt{n} is less than 32, proceed to the byte search part of
  the algorithm (\texttt{.found}).
  \item If \texttt{n} is 32 or greater, can use the SVP64 Horizontal-First
  routine, \texttt{.inner}. If this is the case, do the preamble
  before \texttt{.inner}:
  \begin{enumerate}
    \item Check if \texttt{n} is a multiple of 8. If not, need to check
    several (\texttt{n} modulo 8) tail characters of the string before the
    SVP64 routine can be used. The check routine is the one as was used in
    \texttt{.found} (for both \texttt{memchr} and \texttt{memrchr}).
    \item Another adjustment to \texttt{in\_ptr} is needed (decrement pointer
    by 7) at the start of the function, because 8 bytes are being read at
    a time in the SVP64 routine.
    \item First store value of \acrshort{CTR} in \texttt{ctr}.
    Configure the \acrshort{CTR}. Additional \texttt{tmp} register stores
    \texttt{ctr+1} (needed to make sure the vectorised branch will loop
    \texttt{ctr} times).
  \end{enumerate}
\end{enumerate}

\subsubsection{Inner loop}

Comprises the inner loop of the SVP64 Horizontal-First algorithm.

\begin{itemize}
  \item Use 4 consecutive registers (starting at \texttt{addr0} or
  \texttt{GPR[12]}) to store \texttt{(in\_ptr)}, \texttt{(in\_ptr)-8},
  \texttt{(in\_ptr)-16}, \texttt{(in\_ptr)-24} which will be used for
  the Horizontal-First routine.
  \item The instructions \texttt{sv.ld}, \texttt{sv.cmpb}, \texttt{sv.cmpi},
  \texttt{sv.bc} function similarly to the \texttt{memchr} implementation,
  but due to the Horizontal-First mode, are done as separate batches.
  On a match, branch to \texttt{.determine\_loc}.
  \item If no matches have occurred, move string pointer back by 32 (as the
  code processed 32 bytes) and decrease \texttt{n} by 32.
  \item Branch if \texttt{n} is now less than 32. This is required because
  \texttt{in\_ptr} needs to be incremented by 7 (as the code will now be
  loading bytes, not an 8-byte word. Otherwise branch back to \texttt{.outer}
  to determine how to continue.
\end{itemize}

\subsubsection{Determine location}

Use the current value of CTR to calculate which 8-byte block of chars contains
the matching character. \texttt{in\_ptr} is adjusted accordingly.

\subsubsection{Move forward by 7}

At the start of \texttt{memrchr} function call, \texttt{in\_ptr} was
moved back by 7 (to correctly read 8 bytes). Now \texttt{in\_ptr} has to be
moved forward by 7 (because the code will now search by byte starting at the
later byte).

\subsubsection{Found loop}

Almost identical to the \texttt{memchr} version, except \texttt{in\_ptr} is
decremented (since \texttt{memrchr} searches from the end).

\subsubsection{Post-analysis issues discovered}

\begin{itemize}
  \item With more time (and better understanding of the \acrshort{SVP64}), plenty of
  improvements can be made.
  \item The branch conditional target address is manually calculated, requiring
  to look at the objdump of the assembler to determine the correct offset.
  This is due to binutils calculating the wrong address. The procedure for this
  can be found in Section \ref{subsec:determine_cause_err}
  on page \pageref{subsec:determine_cause_err}.
\end{itemize}


\clearpage
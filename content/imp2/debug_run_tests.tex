\section{Debugging and Running Function Tests}

\subsection{Running tests and checking if errors present}

Figuring out the issue with a function is an iterative process, and may require
looking at many different files.\\

For this guide, \texttt{strchr} will be used (as it produced errors at the time).\\

To start with, run compile and run the test cases:

\begin{verbatim}
(glibc-svp64)$: cd ~/src/glibc-svp64/svp64-port/
(glibc-svp64)$: make clean
(glibc-svp64)$: make all
(glibc-svp64)$: SILENCELOG='!instr_in_outs' ./test-strchr-svp64 --direct \
                >& /tmp/f
\end{verbatim}

(The \texttt{'!instr\_in\_outs'} setting for the \texttt{SILENCELOG}
environment variable reduces the logging coming out of ISACaller,
which usually isn't required for day-to-day testing. Full logging
however is useful when trying to determine if the instruction
behaviour is correct.)\\

The test results can be found in the temporary file \texttt{f} under
\texttt{/tmp/}. The file can be opened and inspected while the test regression
is running (although it's best to use a lightweight text editor like
\texttt{vim} as the file will grow quickly).\\

To check if there were any failures without opening the test result file:

\begin{verbatim}
(glibc-svp64)$: echo $?
\end{verbatim}

The value of the \texttt{?} variable will be non-zero if there are errors. Test
regression failures are caused by a mismatch between the reference C
implementation of \texttt{strchr} and the Simple-V implementation.\\

Inside the result tests file for the occurance of
\texttt{Wrong result in function}.
The version of \texttt{strchr\_svp64} used for this example is in git commit
\texttt{\#9378006a}. Simple-V implementation
\href{https://git.vantosh.com/ngisearch/glibc-svp64/src/commit/9378006a84bdef6af85eb0f810fb62fedc62c588/svp64-port/svp64/strchr_svp64.s}{strchr\_svp64.s}.
Three errors were found in the first group of tests,
\begin{verbatim}
do_test (0, 16 << i, MAX_SIZE, SMALL_CHAR, MIDDLE_CHAR);
\end{verbatim}

See,
\href{https://git.vantosh.com/ngisearch/glibc-svp64/src/commit/9378006a84bdef6af85eb0f810fb62fedc62c588/svp64-port/test-strchr.c#L267}{test-strchr.c\#267}.

\subsection{Determining causes of errors}
\label{subsec:determine_cause_err}

\subsubsection{Incorrect value for character \textbf{c} loaded into \texttt{GPR[4]}}

Looking at the intial debug printout, one can see that the give byte/character
\textbf{c} is supposed to be \texttt{0xfe}, however the value being printed
is \texttt{ffffffff}, while the value being stored in register \texttt{GPR[4]}
is \texttt{0xfffffffffffffffe}.

\begin{verbatim}
strchr called: s: 0x7ffd93fd8ad0, c: fe
strchr_svp64 called: s: 0x7ffd93fd8ad0, c: fe
homeIsaDir: /home/am/src/openpower-isa/src/openpower/decoder/isa/
bytes: 0, bytes_rem: 1
ffffffff
binary <_io.BytesIO object at 0x7f74e7e3f0a0>

Memory:
0x00100000:  FF 00 00 00 00 00 00 00  00 00 00 00 00 00 00 00

call ori ori
read reg 4/0: 0xfffffffffffffffe
read reg 8/0: 0x0
write reg r(8, 8, 0) 0xfffffffffffffffe ew 64
reg  0 00000000 00000000 00000000 00100000 fffffffffffffffe ...
reg  8 fffffffffffffffe 00000000 00000000 00000000 00000000 ...
\end{verbatim}

Both wrapper and test C files were fixed to pass in a single byte, see
\href{https://git.vantosh.com/ngisearch/glibc-svp64/commit/8f1b25340ee2f108027a6f50e365d42aeb7cc939}{commit}.

\subsubsection{Wrong return value}

The failed test in question is the second one in the test regression. Can be
found by searching for the second occurance of the term \texttt{Memory}
(shows the full string to be tested loaded in memory, line \texttt{\#48602}),
and by searching for the first occurance of \texttt{Wrong} term
(the end of the test, line \texttt{\#48510}).\\

(Given line numbers based on \texttt{SILENCELOG='!instr\_in\_outs'} value.)\\

Start of test:

\begin{verbatim}
strchr_svp64 called: s: 0x7ff622cbd000, c: 17(^W)
bytes: 256, bytes_rem: 0
\end{verbatim}

String loaded into memory (only first 48 bytes shown):

\begin{verbatim}
    Memory:
    0x00100000:  20 37 4E 65 7C 93 2B 42  59 70 87 9E 36 4D 64 7B
    0x00100010:  92 2A 41 58 6F 86 9D 35  4C 63 7A 91 29 40 57 6E
    0x00100020:  17 9C 34 4B 62 79 90 28  3F 56 6D 84 9B 33 4A 61
    0x00100030:  ...
\end{verbatim}

End of test:

\begin{verbatim}
return val        : 0000000000000000
./test-strchr-svp64: Wrong result in function
                     strchr_svp64 0x17 (nil) 0x7ff622cbd020
\end{verbatim}

The return value should be \texttt{100020}, because that's where
the char \texttt{0x17} occurs.\\

After looking through the instruction listings and register values in the
results file, it was discovered that the \texttt{sv.bc} branch address
was incorrect.\\

Place of interest starts at line \texttt{\#26887}, most empty register
lines ommitted).

\begin{verbatim}
call mtspr mtspr
read reg 6/0: 0x5
reg  0 00000000 00000000 00000000 00100020 00000017 000000e0 00000005 00000004
reg  8 1717171717171717 0000 0000 0000 00100020 00100028 00100030 00100038
reg 16 289079624b349c17 614a339b846d563f 9a836c553e278f78 543d268e77604932 0
... skip reg values ...
call bc bc
write reg CTR 0x4
reg  0 00000000 00000000 00000000 00100020 00000017 000000e0 00000005 00000004
reg  8 1717171717171717 0000 0000 0000 00100020 00100028 00100030 00100038
reg 16 289079624b349c17 614a339b846d563f 9a836c553e278f78 543d268e77604932 0
... skip reg values ...
call subf subf
read reg 7/0: 0x4
read reg 6/0: 0x5
read reg 6/0: 0x5
write reg r(6, 6, 0) 0x1 ew 64
reg  0 00000000 00000000 00000000 00100020 00000017 000000e0 00000001 00000004
reg  8 1717171717171717 0000 0000 0000 00100020 00100028 00100030 00100038
reg 16 289079624b349c17 614a339b846d563f 9a836c553e278f78 543d268e77604932 0
... skip reg values ...
\end{verbatim}

The \texttt{mtspr} instruction corresponds to \texttt{mfspr}
(an extended mnemonic) listing,
\href{https://git.vantosh.com/ngisearch/glibc-svp64/src/commit/9378006a84bdef6af85eb0f810fb62fedc62c588/svp64-port/svp64/strchr_svp64.s#L108}{line \#108},
of the assembly code.\\

The correct behaviour is on a successful byte match (in this case
matches \texttt{0x17} as expected) there should be a branch to line
\texttt{\#108}, however the simulator jumps to the \texttt{subf}
(assembly using \texttt{sub} extended mnemonic),
\href{https://git.vantosh.com/ngisearch/glibc-svp64/src/commit/9378006a84bdef6af85eb0f810fb62fedc62c588/svp64-port/svp64/strchr_svp64.s#L112}{line \#112}.\\

Turns out the manual \texttt{sv.bc} branch address wasn't update after
copying assembler from existing function.\\

Due to binutils not fully supporting \texttt{sv.bc}, an incorrect
branch address is generated when a label is given in the assembly file.\\

For now, the branch address must be calculated manually.
The method for doing this:

\begin{itemize}
  \item Run the makefile to generate \texttt{strchr\_svp64.o}.
  \item Generate an assembly listing.
\end{itemize}

Command to do so:

\begin{verbatim}
(glibc-svp64)$: powerpc64le-linux-gnu-objdump -D svp64/strchr_svp64.o \
                > svp64/dump_strchr.txt
\end{verbatim}


\begin{itemize}
  \item Open the file and look at where the \texttt{sv.bc} instruction starts.
\end{itemize}

The dump section in question:

\begin{verbatim}
0000000000000054 <.inner>:
  54:   78 1b 6c 7c     mr      r12,r3
  ... skip instructions ...
  84:   00 20 40 05     .long 0x5402000
  88:   1c 00 02 40     bdnzf   eq,a4 <.determine_loc+0xc>
  ... skip instructions ...

0000000000000098 <.determine_loc>:
  98:   a6 02 e9 7c     mfctr   r7
  9c:   04 00 c0 38     li      r6,4
  a0:   50 30 c7 7c     subf    r6,r7,r6
  a4:   00 00 26 2c     cmpdi   r6,0
  ... skip instructions ...
\end{verbatim}

The \texttt{sv.bc} instruction starts at address \texttt{0x84}, because
the prefix is included as part of the instruction for the calculation of the
relative branch address.\\

The branch address shown by the object dump (\texttt{0xa4}) is 4 bytes higher
than the actual branch address taken (during simulation, the branch jumps to
where the \texttt{subf} instruction occurs, address \texttt{0xa0}.


\begin{itemize}
  \item Find the desired address to jump to. Looking at the object dump,
  this will correspond to address \texttt{0x98}, or where the
\texttt{.determine\_loc} label is.
  \item Calculate the offset to use by subtracting the start of \texttt{sv.bc}
  from the desired address.
\end{itemize}

Calculation:

\begin{verbatim}
addr_target - addr_sv.bc = 0x98 - 0x84 = 0x14
\end{verbatim}

Thus the \texttt{target\_address} parameter in \texttt{sv.bc}
(last argument) needs to be changed to {0x14}.

The code has been fixed in
\href{https://git.vantosh.com/ngisearch/glibc-svp64/commit/2d2c0f70dc5cca10a1c5d92d726406903f9e5b23}{commit \#2d2c0f70}.
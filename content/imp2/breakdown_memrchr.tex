% SPDX-License-Identifier: LGPL-3-or-later
% Copyright 2023 VectorCamp
% Copyright 2023 Red Semiconductor Ltd.
% Copyright 2023 Vantosh
%
% Funded by NGI Search Programme HORIZON-CL4-2021-HUMAN-01 2022,
% https://www.ngisearch.eu/, EU Programme 101069364.
\section{\texttt{memrchr} Function Breakdown}

\subsection{What is \texttt{memrchr}?}

\begin{verbatim}
void *memrchr(const void *s, int c, size_t n);
\end{verbatim}

This function behaves similarly to \texttt{memchr}, but searches in reverse.
In terms of the implementation, this requires a bit of additional arithmetic
to start looking in reverse.\\

As demonstration of SVP64, a \textit{Horizontal-First implementation is used
instead of Vertical-First}.\\

(Sections similar to \texttt{memchr} are not covered in this section.)

\subsection{\texttt{memrchr\_svp64.s} - SVP64 assembler implementation of \texttt{memrchr}}

Source code for \texttt{memrchr\_svp64.s} can be found in the
\texttt{glibc-svp64} repo, path is:
\begin{verbatim}
svp64-port/svp64/memrchr_svp64.s
\end{verbatim}

\subsubsection{Setup}

In addition to code from \texttt{memchr}, adjust \texttt{in\_ptr}
to start at N-1 (the last char of the string).

\subsubsection{Outer loop}

\begin{enumerate}
  \item Must determine if number of remaining chars, \texttt{n} of
  string is zero. If so, return null (no match found).
  \item If \texttt{n} is less than 32, proceed to the byte search part of
  the algorithm (\texttt{.found}).
  \item If \texttt{n} is 32 or greater, can use the SVP64 Horizontal-First
  routine, \texttt{.inner}. If this is the case, do the preamble
  before \texttt{.inner}:
  \begin{enumerate}
    \item Check if \texttt{n} is a multiple of 8. If not, need to check
    several (\texttt{n} modulo 8) tail characters of the string before the
    SVP64 routine can be used. The check routine is the one as was used in
    \texttt{.found} (for both \texttt{memchr} and \texttt{memrchr}).
    \item Another adjustment to \texttt{in\_ptr} is needed (decrement pointer
    by 7) at the start of the function, because 8 bytes are being read at
    a time in the SVP64 routine.
    \item First store value of \acrshort{CTR} in \texttt{ctr}.
    Configure the \acrshort{CTR}. Additional \texttt{tmp} register stores
    \texttt{ctr+1} (needed to make sure the vectorised branch will loop
    \texttt{ctr} times).
  \end{enumerate}
\end{enumerate}

\subsubsection{Inner loop}

Comprises the inner loop of the SVP64 Horizontal-First algorithm.

\begin{itemize}
  \item Use 4 consecutive registers (starting at \texttt{addr0} or
  \texttt{GPR[12]}) to store \texttt{(in\_ptr)}, \texttt{(in\_ptr)-8},
  \texttt{(in\_ptr)-16}, \texttt{(in\_ptr)-24} which will be used for
  the Horizontal-First routine.
  \item The instructions \texttt{sv.ld}, \texttt{sv.cmpb}, \texttt{sv.cmpi},
  \texttt{sv.bc} function similarly to the \texttt{memchr} implementation,
  but due to the Horizontal-First mode, are done as separate batches.
  On a match, branch to \texttt{.determine\_loc}.
  \item If no matches have occurred, move string pointer back by 32 (as the
  code processed 32 bytes) and decrease \texttt{n} by 32.
  \item Branch if \texttt{n} is now less than 32. This is required because
  \texttt{in\_ptr} needs to be incremented by 7 (as the code will now be
  loading bytes, not an 8-byte word. Otherwise branch back to \texttt{.outer}
  to determine how to continue.
\end{itemize}

\subsubsection{Determine location}

Use the current value of CTR to calculate which 8-byte block of chars contains
the matching character. \texttt{in\_ptr} is adjusted accordingly.

\subsubsection{Move forward by 7}

At the start of \texttt{memrchr} function call, \texttt{in\_ptr} was
moved back by 7 (to correctly read 8 bytes). Now \texttt{in\_ptr} has to be
moved forward by 7 (because the code will now search by byte starting at the
later byte).

\subsubsection{Found loop}

Almost identical to the \texttt{memchr} version, except \texttt{in\_ptr} is
decremented (since \texttt{memrchr} searches from the end).

\subsubsection{Post-analysis issues discovered}

\begin{itemize}
  \item With more time (and better understanding of the SVP64), plenty of
  improvements can be made. That said, limited support for advanced SVP64
  features in binutils puts a limitation on which features can be used.
  \item The branch conditional target address is manually calculated, requiring
  to look at the objdump of the assembler to determine the correct offset.
  This is due to binutils calculating the wrong address. The procedure for this
  can be found in Section \ref{subsec:determine_cause_err}
  on page \pageref{subsec:determine_cause_err}.
\end{itemize}

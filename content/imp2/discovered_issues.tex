% SPDX-License-Identifier: LGPL-3-or-later
% Copyright 2023 Red Semiconductor Ltd.
%
% Funded by NGI Search Programme HORIZON-CL4-2021-HUMAN-01 2022,
% https://www.ngisearch.eu/, EU Programme 101069364.

\chapter{Issues Discovered During Development}

\section{Incorrect Next Instruction Address (NIA) in `sv.bc` branch instruction}

\begin{itemize}
  \item Raised bug on the Libre-SOC bug tracker:
  \href{https://bugs.libre-soc.org/show_bug.cgi?id=1210}{bug \#1210}
  \item Libre-SOC commit \#d9544764 which fixed this issue:
  \href{https://git.libre-soc.org/?p=openpower-isa.git;a=commitdiff;h=d9544764b1710f3807a9c0685d150a665f70b9a2}{here}
  \item Libre-SOC unit test which demonstrated this issue:
  \href{https://git.libre-soc.org/?p=openpower-isa.git;a=blob;f=src/openpower/decoder/isa/test_caller_svp64_bc.py;h=93689ded619f8fa67b455f18b122fa60220ddea1;hb=089e6d352ec57be4ab645d18ad9e95df3af0d365#l310}{here}
\end{itemize}

The \acrshort{SVP64} implementation of \texttt{memchr} uses the vectorised branch
conditional instruction, \texttt{sv.bc} in the \acrshort{SVP64} Vertical-First
\href{https://git.vantosh.com/ngisearch/glibc-svp64/src/commit/1afb94889b8ea2f85844e410f87e5a9b8e2e959f/svp64-port/svp64/memchr_svp64.s#L67}{loop}.\\

The instruction below is the Simple-V form of the normal \texttt{bc} instruction.
It works by checking a single bit in the \acrfull{CR}
(see Section 2.3.1 of Book I, PowerISA v3.0C spec for more information
on \acrshort{CR}).
The branch is taken depending on the set \texttt{BO} mode (which may also
involve the \acrfull{CTR} (Section 2.3.3) and the CR bit specified.

\begin{verbatim}
sv.bc               0, *2, .found
\end{verbatim}

Has a \texttt{BO=0} mode which means "Decrement the CTR, then branch
if the decremented CTRM:63 $\neq$ 0 and CRBI=0". (See Figure 40, Section 2.4
Branch Instructions for the full list of \texttt{BO} modes.).\\

\texttt{CR} is split into 8 4-bit CR Fields 0-7, where every 4-bits correspond
to (copied from Section 2.3.1 of PowerISA spec):

\begin{itemize}
  \item 0: Negative (LT), The result is negative.
  \item 1: Positive (GT), The result is positive.
  \item 2: Zero (EQ), The result is zero.
  \item 3: Summary Overflow (SO)
\end{itemize}

So this instruction is supposed to branch only \textit{if}:

\begin{itemize}
  \item CTR value after decrementing is greater than zero
  \item EQ bit of CR Field 0 (bit 2+32 of CR) is zero.
At every consecutive iteration of the Simple-V loop (in this case,
the Simple-V index is going up from 0 to (VL-1)), the CR Field will move up
by 1 (i.e. CR Field 0, CR bit 2+32; CR Field 1, CR bit 6+32;
CR Field 2, CR bit 10+32; CR Field 3, and so on).
\end{itemize}

In the context of \texttt{memchr}, the branch must occur if one of the 8 bytes
in registers \texttt{s} (starts at \texttt{GPR[16]} increases up to
\texttt{`GPR[19]}) contains the matching byte \texttt{c}
(in \texttt{GPR[4]}).\\

Before Libre-SOC's ISACaller simulator was fixed, the following would occur:
the \texttt{CR} bit will be set correctly by the compare immediate instruction
\texttt{sv.cmpi}, but during the \texttt{sv.bc} instruction, the simulator
will proceed to the next instruction address (\texttt{svstep})
instead of branching to \texttt{.found}.

\subsubsection{Replicating this issue}

It is assumed the chroot environment has already been setup.

(Using the commit \texttt{\#089e6d35} of the \texttt{openpower-isa} repo:
\href{https://git.libre-soc.org/?p=openpower-isa.git;a=commitdiff;h=089e6d352ec57be4ab645d18ad9e95df3af0d365}{here}).

\begin{verbatim}
(glibc-svp64)$: cd ~/src/openpower-isa
(glibc-svp64)$: git checkout 089e6d352ec57be4ab645d18ad9e95df3af0d365
(glibc-svp64)$: make
\end{verbatim}

Running \texttt{make} will regenerate the necessary files for the simulator.

To run the standalone unit test written to demonstrate the issue:
\begin{verbatim}
(glibc-svp64)$: cd ~/src/openpower-isa/src/openpower/decoder/isa
(glibc-svp64)$: python3 test_caller_svp64_bc.py >& /tmp/f
\end{verbatim}

The test is called \texttt{test\_sv\_branch\_vertical\_first()}. Other tests inside
\texttt{test\_caller\_svp64\_bc.py} could be disabled by changing the
\texttt{test\_} prefix to something else, and they will be ignored.

In the results file, the following message will be present:
\begin{verbatim}
FAIL: test_sv_branch_vertical_first (__main__.DecoderTestCase)
this is a branch-vertical-first-loop demo which shows an early
----------------------------------------------------------------------
Traceback (most recent call last):
  File "test_caller_svp64_bc.py", line 341, in test_sv_branch_vertical_first
    self.assertEqual(sim.spr('CTR'), SelectableInt(1, 64))
AssertionError: SelectableInt(value=0x0, bits=64) !=
                SelectableInt(value=0x1, bits=64)
\end{verbatim}

And the reason for the incorrect \acrshort{CTR} value, is because instead of the expected
branch, another iteration of the \acrfull{SV} loop occurs, thus
decrementing \acrshort{CTR} to 0.

To run the \texttt{memchr} tests:
\begin{verbatim}
(glibc-svp64)$: cd ~/src/glibc-svp64/svp64-port/
(glibc-svp64)$: make clean
(glibc-svp64)$: make all
(glibc-svp64)$: SILENCELOG='!instr_in_outs' ./test-memchr-svp64 --direct \
                >& /tmp/f
\end{verbatim}


Searching for the first occurance of \texttt{Wrong}, the following is found on
\texttt{line \#31233}:
\begin{verbatim}
return val        : 0000000000000000
./test-memchr-svp64: Wrong result in function memchr_svp64 (nil) 0x7f2abc182020
\end{verbatim}

The returned value from the \acrshort{SVP64} implementation was 0, while the reference C
gave 0x7f2abc182020 (the correct result).\\

Jump to the occurance of \texttt{Memory:} which occurred before the error
message (\texttt{line \#461}). The memory printout (only first 48 bytes shown):
\begin{verbatim}
Memory:
0x00100000:  01 18 2F 46 5D 74 0C 23  3A 51 68 7F 18 2E 45 5C
0x00100010:  73 0B 22 39 50 67 7E 16  2D 44 5B 72 0A 21 38 4F
0x00100020:  17 7D 15 2C 43 5A 71 09  20 37 4E 65 7C 14 2B 42
0x00100030:  ...
\end{verbatim}

(The simulator's memory address space is different from the native C,
but the \texttt{memchr\_wrapper.c} code takes care of the conversion.)\\

To find where the incorrect branch occurred, search for
\texttt{9715a432c157d17} (in reverse order compared to memory print because of
Big/Little-Endian conversion). The first occurance of this value is when
the doubleword load \texttt{ld} is called. This corresponds to
\href{https://git.vantosh.com/ngisearch/glibc-svp64/src/commit/1afb94889b8ea2f85844e410f87e5a9b8e2e959f/svp64-port/svp64/memchr_svp64.s#L64}{line \#64}
of the \texttt{memchr} SVP64 assembler.\\

The instructions called after \texttt{ld} are:

\begin{itemize}
  \item Compare bytes \texttt{sv.cmpb} (generates the mask \texttt{0xff}
        because \texttt{0x17} is the character to match for)
  \item Compare immediate \texttt{sv.cmpi} (sets bit 1 (GT), and clears
        bit 2 (EQ), of \acrshort{CR} Field 0)
  \item Branch conditional (decrements \acrshort{CTR}, tests for bit2=0, but
        doesn't branch)
  \item \acrshort{SV} step \texttt{svstep} (branch was not taken,
        \textit{this shouldn't have happened})
\end{itemize}

\subsubsection{Debugging with Python debugger}

Add the following lines to \texttt{caller.py} in the
`~/src/openpower-isa/src/openpower/decoder/isa/` directory.
\begin{verbatim}
import pdb; pdb.set_trace() # Add at line #15
breakpoint()                # Add at line #2361 inside the call() method
\end{verbatim}

The breakpoint in \texttt{caller.py} is not necessary at first, as it will
stop every time an instruction is called (it will take a while until the
above scenario occurs).\\

The main place where breakpoint should be set is in:

\begin{verbatim}
~/src/openpower-isa/src/openpower/decoder/isa/generated/svbranch.py
\end{verbatim}

after \texttt{line \#18}. This breakpoint will stop at
every call of \texttt{sv.bc}.\\

To run:
\begin{verbatim}
(glibc-svp64)$: cd ~/src/glibc-svp64/svp64-port/
(glibc-svp64)$: SILENCELOG='!instr_in_outs' ./test-memchr-svp64 --direct
\end{verbatim}

The Python intepreter will stop at the very beginning of \texttt{caller.py}.
Enter \texttt{c} (continue). The intepreter will stop at the first instance
of \texttt{sv.bc}. Continue until the memory address in register
\texttt{GPR[3]} is equal to \texttt{00100020}.
After entering \texttt{c} another 4 times, the intepreter will stop at the
call where the branch doesn't occur.\\

Now the python code can be single-stepped using \texttt{n} (next) and
\texttt{s} (step, steps into functions instead of just running
and returning result).\\

The pseudo-code for \texttt{sv.bc} can be stepped through,
confirming that all the necessary conditions for the branch occur.
Compare with the pseudo-code found in
`~/src/openpower-isa/openpower/isa/svbranch.mdwn`,
\href{https://git.libre-soc.org/?p=openpower-isa.git;a=blob;f=openpower/isa/svbranch.mdwn;h=e8b46e7700b44c6112ee2d873cc2e04b3c732370;hb=089e6d352ec57be4ab645d18ad9e95df3af0d365}{repo},
also mirrored on the
\href{https://libre-soc.org/openpower/isa/svbranch/}{Libre-SOC wiki}.\\

The generated pseudo-code function correctly updates the \acrfull{NIA},
but it is later overwritten by the call to 
\begin{verbatim}
yield from self.do_nia(asmop, ins_name, rc_en, ffirst_hit)
\end{verbatim}

On \texttt{line \#2362} of \texttt{caller.py}. For more details of the fix,
please see the Libre-SOC
\href{https://bugs.libre-soc.org/show_bug.cgi?id=1210}{bug 1210}.

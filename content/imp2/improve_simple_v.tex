% SPDX-License-Identifier: LGPL-3-or-later
% Copyright 2023 Red Semiconductor Ltd.
%
% Funded by NGI Search Programme HORIZON-CL4-2021-HUMAN-01 2022,
% https://www.ngisearch.eu/, EU Programme 101069364.

\chapter{Improving Simple-V spec}

\begin{enumerate}
  \item To check and improve Libre-SOC's current Simple-V spec by finding/implementing
  new features relevant to processing strings and search algorithms in general
  (as part of \acrshort{NGI} Search grant goals).
\end{enumerate}

An example feature which came out of this research is
Data-Dependent Fail-First, which allows to iterate over data (in
this context, an array of byte characters) and terminate early when
a condition is met (hence "fail-first"), and shorten the specified
vector length based on this early termination. This feature is very useful
for string search and copy functions.
One great example is the \texttt{strncpy} function,
code \href{https://git.libre-soc.org/?p=openpower-isa.git;a=blob;f=src/openpower/decoder/isa/test_caller_svp64_ldst.py;h=4ecf534777a5e8a0178b29dbcd69a1a5e2dd14d6;hb=d9544764b1710f3807a9c0685d150a665f70b9a2#l31}{here}
\footnote{Libre-SOC \texttt{openpower-isa} repo commit \texttt{\#d9544764...}
(\texttt{fix vertical-first sv.bc}) is used to refer to
\acrshort{SVP64} spec and code examples. Commit link
\href{https://git.libre-soc.org/?p=openpower-isa.git;a=commit;h=d9544764b1710f3807a9c0685d150a665f70b9a2}{here}.}

\section{Simple-V assistance to VectorCamp and VanTosh}

\begin{enumerate}
  \item Provide assistance with currently implemented features of Simple-V.
  \item To find causes and/or workarounds with code bugs due to the ISACaller
simulator (the simulator which supports PowerISA 3.0 and implements the current
Simple-V specification).
  \begin{enumerate}
    \item If a bug related to the simulator is found, to create a test
  case and submit a bug report to Libre-SOC (see the \texttt{sv.bc} branching address
  bug in \href{https://bugs.libre-soc.org/show_bug.cgi?id=1210}{bug \#1210}
  as an example).
    \item Document current workarounds required due to binutils support lagging
  behind current Simple-V (as a result of time/budget constraints in Libre-SOC):
    \begin{enumerate}
      \item  Use of original assembler instruction required instead of
      extended mnemonics when using Simple-V \texttt{sv.} prefix.
      Examples of extended mnemonics are \texttt{nop}, \texttt{bdnz},
      \texttt{subi}, etc.). Instead have to use
      \texttt{ori 0,0,0}, \texttt{bc 16,0,target}, \texttt{addi Rx,Ry,-value}.
    \end{enumerate}
    \item \texttt{sv.bc} with \texttt{BO=0} (other \texttt{BO} modes weren't tested, see
  Figure 40. BO field encodings in section 2.4 of PowerISA 3.0 Book I
  for the full list), has to be given a manually computed address.
  See an example of how to do this in Section \ref{subsec:determine_cause_err}
  on page \pageref{subsec:determine_cause_err}.
  \end{enumerate}
\end{enumerate}

\section{Standard C function Simple-V implementations}

To write implementations of several \texttt{glibc} string functions utilising a subset
of the available Simple-V feature set (due to limited binutils support).

\section{Integration into the \texttt{glibc} test environment}

We did the initial work of studying \texttt{glibc},
creating wrapper code, and writing up an example function, \texttt{memchr()},
after which we could use that as a template (from which the other functions were
written up).

\subsubsection{Documentation}

We have planned to document several critical parts of the developments for:
reproducibility; to aid understanding; and help developers to come back to
the code in future milestone and/or projects.
\par

The documentation includes:

\begin{itemize}
  \item Environment setup
  \item Running existing test regressions and debugging
  \item Creating new function implementations
  \item Document discovered bugs in instruction simulator
\end{itemize}

\section{Moving Towards IMP3}

Deliverable during \acrshort{IMP}3 will be to create a report on how \acrshort{SVP64} optimised
search in the context of Vectorscan and libc6. The work done so far in \acrshort{IMP}2
has taught all participants more about the theoretical capabilities
of \acrshort{SVP64}, as well as how to create \acrshort{SVP64} programs. The achievements and
shortcomings will be documented, allowing the Simple-V specification to
be adapted and improved accordingly.

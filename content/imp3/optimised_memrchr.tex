% SPDX-License-Identifier: LGPL-3-or-later
% Copyright 2023 VectorCamp
% Copyright 2023 Red Semiconductor Ltd.
% Copyright 2023 VanTosh
%
% Funded by NGI Search Programme HORIZON-CL4-2021-HUMAN-01 2022,
% https://www.ngisearch.eu/, EU Programme 101069364.

\section{Optimised \texttt{memrchr} Function}

\subsection{The code}

% TODO: Use Github link
The function was implemented within Python framework can be found
\href{https://github.com/ngisearchsvp64/glibc-svp64/blob/master/svp64-port/experimentation/test_memrchr.py}{here}.

The start and end bits of code have been skipped (same as \texttt{memchr}).
\begin{verbatim}
  mtspr 9, 5 # move r5 to CTR
  add 7,3,5 # (start address+len-1)
  addi 7,7,-1 # Now points to last address of array s
  # VL (and r1) = MIN(CTR,MAXVL)
  setvl 1,0,4,0,1,1" # maxvl=4
  # load VL bytes (update r7 addr, current pointer)
.inner:
  sv.lbzu/pi *16, 0xffff(7)
  # cmp against zero, truncate VL
  sv.cmp/ff=eq/vli *0,1,*16,4
  # test CTR, stop if any cmp failed
  sv.bc/all 0, *2, .inner # -0x10

  # Check for no match, add offset to get actual found address
  # If pointer is less than starting address 3, no match.
  cmp 0,1,7,3
  bc 12, 0, .no_match # 0x14, check LT bit, if set, branch

  # Adjust pointer in Reg7 because search ends at multiples of
  # VL value
  setvl 6,0,1,0,0,0 # Get current value of VL
  subfic 6,6,5" # calculate offset for found char - (maxvl+1)
  # Reg 3 will now be found address, or one byte outside of array
  add 3,7,6
  b .end # 0x8
\end{verbatim}

\subsection{Code explanation}
Only the differences between \texttt{memchr} and \texttt{memrchr} are going to
be discussed

\begin{verbatim}
  add 7,3,5 # (start address+len-1)
  addi 7,7,-1 # Now points to last address of array s
\end{verbatim}
\acrshort{GPR} \#7 is stores the current address to load from. As \texttt{memrchr}
begins from the last byte of array, the number of bytes \texttt{n} needs to be
added. However since addresses begin from 0, 1 must be subtracted to get the
last byte.

\begin{verbatim}
.inner:
  sv.lbzu/pi *16, 0xffff(7)
  # cmp against zero, truncate VL
  sv.cmp/ff=eq/vli *0,1,*16,4
  # test CTR, stop if any cmp failed
  sv.bc/all 0, *2, .inner # -0x10
\end{verbatim}
There is one main difference between \texttt{memchr} and \texttt{memrchr} in
this portion of code. Instead of incrementing the address after loading,
the code decrements instead. The Python SVP64Asm class didn't accept
\texttt{-1} argument, so \texttt{0xFFFF} is used to represent \texttt{-1}
in 16-bit 2's complement (\texttt{lbzu} takes a 16-bit signed integer
as an immediate).

\begin{verbatim}
# Check for no match, add offset to get actual found address
# If pointer is less than starting address 3, no match.
cmp 0,1,7,3
bc 12, 0, 0x14 # check LT bit, if set, branch
\end{verbatim}

Compare the current address at \acrshort{GPR} \#7 with the starting address.
If the current address is smaller (less than,
LT Condition Register field bit 0), that means no match occurred.
In such a case jump to \texttt{.no\_match}.

\begin{verbatim}
# Adjust pointer in Reg7 because search ends at multiples of
# VL value
setvl 6,0,1,0,0,0 # Get current value of VL
subfic 6,6,5" # calculate offset for found char - (maxvl+1)
# Reg 3 will now be found address, or one byte outside of array
add 3,7,6
b .end # 0x8
\end{verbatim}
This snippet of code is similar except for the offset (since the bytes are
being accessed in reverse).
Also the current address is in \acrshort{GPR} \#7, not \#3.


\chapter{Commentary and Conclusions}

\acrfull{SV} (\acrshort{SVP64} being a particular variant for \acrshort{POWER}
ISA) is a powerful extension on top of scalar instructions, if you know the set
of features it contains, and how to use it.

To demonstrate search pattern algorithms using \acrshort{SVP64}, it was decided
to integrate code written in this extension into the \texttt{glibc}
test framework. This exercise has taught us that \acrshort{SV} still requires
further development.

\section{Red Semiconductor Perspective}
In particular, hardware considerations need to be thought through.
Although it is good to abstract away hardware specifics from the programmer,
it is still necessary to consider how many of \acrshort{SV}'s features can
actually be implemented in reality.

Many advanced features of SVP64, such as the REMAP system, fail-first,
predication, element-width overrides etc., are made difficult to use
by the fact that the reference simulator for the SVP64 specification is
highly experimental, and doesn't always align with the available documentation.
A lot of time was spent during Milestones 2 and 3 simply getting around the
issues of the simulator, and the available assembler notation, and this limited
the scope of function complexity and optimisation.

\section{VectorCamp}

Existing code base for many projects utilising vector extensions are on the
order of 100k lines of code or more.

Thus it is unreasonable to refactor the entire codebase using a new vector
extension, regardless of how performant it may be.

Instead, it is better to gradually upgrade parts of the code (just as is
happening today with the Rust programming language), by substituting specific
vector extension (AVX, NEON, etc.) functions with optimised SVP64 code instead.

This is why during Milestone 3, we went through the exercise of researching
common SIMD operations, and implementing them in SVP64. By having
implementations of standard operations (such as shuffle, movemask, etc.) will
allow gradual refactoring.

However, due to the lack of specific byte manipulation instructions, and
difficulty in configuration setup of the REMAP system, these implementations
turned to be either much bigger than desired, or only partially implemented.

\section{Future Work}

Based on the experience of the consortium, Red has learned enough to move from
the purely theoretical towards a real functional HW implementation.
In practice, this has caused us to have a fundamental rethink of the concepts
and going back to the original theoretical ideas we're now developing a hardware
solution, different from \acrshort{SV}, that delivers equivalent and
advanced functionality.

In this, the \acrshort{NGI} Search process has been informative and educational
as to future developments (which may be the subject of future grant applications).

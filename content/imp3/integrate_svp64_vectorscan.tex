% SPDX-License-Identifier: LGPL-3-or-later
% Copyright 2023 VectorCamp
% Copyright 2023 Red Semiconductor Ltd.
% Copyright 2023 VanTosh
%
% Funded by NGI Search Programme HORIZON-CL4-2021-HUMAN-01 2022,
% https://www.ngisearch.eu/, EU Programme 101069364.
\section{Integrating \acrshort{SVP64} into Vectorscan}

Vectorscan has a large codebase, and it heavily depends on \acrshort{SIMD} intrinsics from
\acrshort{ARM}, Intel, PowerISA, etc., so VectorCamp decided on implementing some common
\acrshort{SIMD} functions using \acrshort{SVP64}. These would then be integrated into a separate
branch of Vectorscan.

VectorCamp began with including the \texttt{pypowersim} (wrapper for the
Libre-SOC simulator) into the \texttt{cmake} build system. This requires
setting up Vectorscan within the same chroot environment used to run the SVP64
simulator, which can be difficult given Vectorscan's dependencies on later
versions of C++ (which are not so easily available with the version of Debian
used for the chroot).

Commits covering the changes done to Vectorscan to support \acrshort{SVP64} are available
\href{https://github.com/ngisearchsvp64/vectorscan/commits/feature/svp64-port/}{here}.

\subsection{Setup script}
The shell script for getting the Vectorscan \acrshort{SVP64} test running can be found
\href{https://github.com/ngisearchsvp64/shell-scripts/blob/main/scripts/ngi-search-vectorscan}{here}.

Fully integrating \acrshort{SVP64} into Vectorscan is a huge task, and the work
done under this grant had shown that more time and effort is required for the
simulator to mature, and for the \acrshort{SIMD} intrinsics to be researched
and written in \acrshort{SVP64}.
Due to the difficulties encountered, and time constraints, the work has not
been fully completed.
There is however a path for the future when/if further funding
opportunities become available.

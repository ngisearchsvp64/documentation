% SPDX-License-Identifier: LGPL-3-or-later
% Copyright 2023 VectorCamp
% Copyright 2023 Red Semiconductor Ltd.
% Copyright 2023 VanTosh
%
% Funded by NGI Search Programme HORIZON-CL4-2021-HUMAN-01 2022,
% https://www.ngisearch.eu/, EU Programme 101069364.

\section{Optimised \texttt{memchr} Function}

\subsection{The code}

% TODO: Use Github link
The function was implemented within Python framework can be found
\href{https://github.com/ngisearchsvp64/glibc-svp64/blob/master/svp64-port/experimentation/test_memchr.py}{here}.
\begin{verbatim}
  cmpi 0,1,5,0 # Check if n==0
  bc 12, 2, .no_match # 0x44
  mtspr 9, 5 # move r5 to CTR
  add 7,3,5 # start address+len
  # start + len + 2 (if this is final pointer val, no match)
  addi 7,7,2
  setvl 1,0,4,0,1,1
  # load VL bytes (update r3 addr, current pointer)
.inner:
  sv.lbzu/pi *16, 1(3)
  # cmp against zero, truncate VL
  sv.cmp/ff=eq/vli *0,1,*16,4
  # test CTR, stop if any cmp failed
  sv.bc/all 0, *2, .inner # -0x10

  # Check for no match, add offset to get actual found address
  # If pointer just outside of array, no match.
  cmp 0,1,3,7"
  bc 12, 2, .no_match # 0x14

  # Adjust pointer in Reg 3 because search ends at multiples of
  # VL value
  setvl 6,0,1,0,0,0", # Get current value of VL
  addi 6,6,-5" # ((-1*maxvl)-1) = (-1*4)-1 = -5, calc offset for found char
  # Reg 3 will now be found address, or one byte outside of array
  add 3,3,6
  b .end # 0x8

.no_match:
  # No match, return null for matched address
  addi 3,0,0
.end:
  # this could be replaced by return call
  nop
\end{verbatim}

\subsection{Code explanation}
\begin{verbatim}
cmpi 0,1,5,0 # Check if n==0
bc 12, 2, .no_match # 0x44
\end{verbatim}
The first part is just a check for $n=0$ case (no need to run function if
there's no array).

\begin{verbatim}
mtspr 9, 5 # move r5 to CTR
\end{verbatim}
Load the length of the byte array from integer \acrfull{GPR} \#5 into
the \acrfull{CTR}, which is used to keep track of loop iterations
(standard use in PowerISA).

\begin{verbatim}
add 7,3,5 # start address+len
# start + len + 2 (if this is final pointer val, no match)
addi 7,7,2
\end{verbatim}
This calculates the address value after the last byte in the array, which is
used to check if no match occurred. This end address is stored in \acrshort{GPR} \#7.

\begin{verbatim}
  setvl 1,0,4,0,1,1
  # load VL bytes (update r3 addr, current pointer)
.inner:
  sv.lbzu/pi *16, 1(3)
  # cmp against zero, truncate VL
  sv.cmp/ff=eq/vli *0,1,*16,4
  # test CTR, stop if any cmp failed
  sv.bc/all 0, *2, .inner # -0x10
\end{verbatim}
This is the main bit of \acrshort{SVP64} code (running in Horizontal-First mode).
The snippet of code will iterate 4 times as specified by immediate
in \texttt{setvl} (checking one byte with each iteration).
Load byte from memory at address stored in \acrshort{GPR} \#3.
The Load Byte and Zero with Update (\texttt{lbzu}) will automatically increment
the address stored in \acrshort{GPR} \#3 with the specified offset (+1 in this case).

The compare instruction will check the loaded byte with the specified char
(stored in \acrshort{GPR} \#4).

A special mode of \acrfull{SV} called \acrfull{FF} is used with
the compare instruction. \acrshort{FF} works by checking given data, and truncating the
\acrfull{VL} if the condition is met. The condition here is that
loaded byte is equal (\texttt{ff=eq}) to the specified byte in \acrshort{GPR} \#4.
The additional \acrshort{VL} inclusive (\texttt{vli}) option ensures the vector is
truncated at the matched byte (since it needs its address is what's required
from \texttt{memchr}).

The conditional branch (\texttt{sv.bc}) in \texttt{BO=0} mode will decrement
\acrshort{CTR}, then branch only if \acrshort{CTR} is non-zero (haven't finished searching the
byte array), \textit{and} the \acrfull{CR} Field 0 bit 2 (EQ) is cleared
(not matching character). For more information on branch instructions,
see Book I, Section 2.4 of the
\href{https://openpower.foundation/specifications/isa/}{PowerISA specification}.

\begin{verbatim}
# Check for no match, add offset to get actual found address
# If pointer just outside of array, no match.
cmp 0,1,3,7
bc 12, 2, .no_match # 0x14
\end{verbatim}

Compare the current address at \acrshort{GPR} \#3 with the address just outside
the byte array for equality. This can only occur if the whole array was
searched, and there was no match. In such a case jump to \texttt{.no\_match}.

\begin{verbatim}
# Adjust pointer in Reg 3 because search ends at multiples of
# VL value
setvl 6,0,1,0,0,0", # Get current value of VL
addi 6,6,-5" # ((-1*maxvl)-1) = (-1*4)-1 = -5, cal offset for found char
# Reg 3 will now be found address, or one byte outside of array
add 3,3,6
b .end # 0x8
\end{verbatim}

When a match occurs, \acrshort{VL} will be truncated to a value between 0 and 3
(since \acrshort{VL} was set to 4 earlier), and for this to be converted to a
final match address, some arithmetic needs to be done.

The current \acrshort{VL} value is stored in a temporary \acrshort{GPR} \#6. Adding it to a
precomputed offset and the current address will produce the correct address
value at \acrshort{GPR} \#3.

\begin{verbatim}
.no_match:
  # No match, return null for matched address
  addi 3,0,0
.end:
  # this could be replaced by return call
  nop
\end{verbatim}

In the case of a no match, \acrshort{GPR} \#3 is cleared (return address set to null)
before being returned.

The main difference between the implementation described in \acrshort{IMP}2
and the one shown above, is that to use the more advanced features of
\acrshort{SVP64}, one has to think more in terms of C-like code (one element at a time),
as opposed to \acrshort{SIMD} (blocks of power-of-2 elements). The initial
implementation processed the byte in chunks of 8-bytes (by loading 8 bytes
from memory into a single 64-bit register), which then required a further
sub-search once a match had been detected.

Although the code size is indeed smaller, the actual throughput (number of
bytes checked per second etc.) is dependent on whether the underlying hardware
supporting \acrshort{SVP64} can recognise the above algorithm and run it in parallel.

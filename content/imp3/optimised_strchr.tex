% SPDX-License-Identifier: LGPL-3-or-later
% Copyright 2023 VectorCamp
% Copyright 2023 Red Semiconductor Ltd.
% Copyright 2023 VanTosh
%
% Funded by NGI Search Programme HORIZON-CL4-2021-HUMAN-01 2022,
% https://www.ngisearch.eu/, EU Programme 101069364.

\section{Optimised \texttt{strchr} Function}

\subsection{The code}

% TODO: Use Github link
The function was implemented within Python framework can be found
\href{https://github.com/ngisearchsvp64/glibc-svp64/blob/master/svp64-port/experimentation/test_strchr.py}{here}.

As \texttt{strchr} is pretty much the same in functionality to \texttt{memchr}
(other than length of string needing to be calculated by \texttt{strchr}), so
only the additional code which calculates the length of the string will be shown.

\begin{verbatim}
  mtspr 9, 5, # move r5 to CTR
  addi 7, 3, 0
  # VL (and r1) = MIN(CTR,MAXVL)
  setvl 1,0,%d,0,1,1 # maxvl=4
  # load VL bytes (update r7 addr, current pointer)
  sv.lbzu/pi *16, 1(7)
  # cmp against zero, truncate VL
  sv.cmpi/ff=eq/vli *0,1,*16,0
  # test CTR, stop if any cmp failed
  sv.bc/all 0, *2, -0x10
  # Adjust pointer in Reg 7 because search ends at multiples of
  # VL value
  setvl 6,0,1,0,0,0 # Get current value of VL
  addi 6,6,-4 # (-1*maxvl), calculate offset for found char
  add 7,7,6
  # Perform (GPR[7]-GPR[3]) to get string length
  subf 5,3,7
\end{verbatim}

After this code, \texttt{memchr} code is used as before as the length
\texttt{n} is now known.
This code is also similar to the main \acrshort{SVP64} code of \texttt{memchr},
except here the loaded byte is compared against the immediate \texttt{0x00},
or a \texttt{NULL} (character used to terminate strings in C).

Vectorised Conditional Branch \texttt{sv.bc} mode \texttt{BO=4} didn't work,
so the example code is limited to strings of length specified in \acrshort{GPR} \#5
(as this register specifies the value of the \acrfull{CTR} used to keep track
of max loop iterations).

\chapter{C optimization}

The C functions for SVP64 are backed by assembly code that we have been able to write and execute in the simulator.
The Assembly code reduces the number of instructions, and while we expect the number of CPU cycles would reduce,
at present we cannot confirm that these reductions would lead to higher speed of execution of code.

The implementation in Assembly for SVP64 routines has proven that the concepts used by Simple-V can be benificial.
However the current implementation does not use all the SVP64 capabilities, but the basic concept demonstrates
that the Vector Looping concept can be benificial for specific algorithms, such as search, cryptography, video
and transactions.
Each of the member companies have a speciality in this field and have determined that the the basic concept of
SVP64 and the Looping mechanism do have a benifit, however without a C abstraction library the wider software
ecosystem would have to rewrite and rethink basic concepts for SVP64 and the looping system.

While a abstraction layer would be a feasable solution and after several conversation within the \acrshort{OPF}
and looking into the ``pveclib" abstraction library, we think that such an endeavour would be possible,
however the grandeur of such a project would require funding and carefull application benefit understanding.

\chapter{SVP64 Power ISA Vector Optimisation for Search}

The goal of the project has been further the development of hardware
optimised for search-related algorithms and string processing by
researching standard C library functionality and creating several
implementations based on experimental \acrshort{SVP64} extension from Libre-SOC.

\section{Introduction}

This project consists of a consortium of three companies represented by:

\begin{itemize}
  \item Red Semiconductor - Andrey Miroshnikov
  \item VectorCamp - Konstantinos Margaritis
  \item VanTosh - Toshaan Bharvani
\end{itemize}

\subsection{What is \acrshort{SVP64} and why is \acrshort{SVP64} relevant to NGI Search?}

\acrfull{SVP64} is a Cray-style vectorisation system which
turns scalar register and scalar instructions into vector operations,
without changing their scalar behaviour. Being an orthogonal extension to the
scalar \acrfull{POWER} ISA subset, it is much smaller and easier to implement in
hardware (simplest implentation being a literal ``for-loop"). \acrshort{SVP64} is
designed to provide the programmer with powerful mathematical operations
without requiring to add dedicated hard-IP blocks (crypto, vector unit, A/V
codecs, etc.).

As for relevance to \acrfull{NGI} Search, it may seem like the Internet is a
Software-only enviroment, in actuality this software must run on hardware.
As a consortium, we are addressing the fundamental performance and
power-effieciency of this hardware, by verifying that low-level memory,
string, regular-expression, and Machine Learning algorithms run optimally
on \acrshort{SVP64}.

The biggest difference between other vector extensions provided by Intel,
ARM, etc., is that if we find \acrshort{SVP64} to be sub-optimal, an opportunity is
provided to us by \acrshort{NGI} Search to investigate how to actually improve the
\acrshort{SVP64} extension to \acrshort{POWER} ISA.

By doing this research, not only \acrshort{SVP64} benefits, but \textit{everyone's} algorithms
benefit from increased performance, and reduced energy requirements.

\subsection{Who are the individual consortium members, and what are their specialisations?}

\begin{itemize}
  \item \textbf{Red Semiconductor} is a fabless semiconductor company which will implement
  working Silicon deploying the \acrshort{SVP64} enhancements to the \acrshort{POWER} ISA, making
  developer kits available to prospective \acrfull{IoT}/edge customers, the
  \acrfull{FOSSHW} and NGI hardware communities
  alike. \acrshort{SVP64} itself is a \acrshort{POWER} ISA version of the general Simple-V
  extension, initially developed by the Libre-SOC project with NLnet funding.
  \item \textbf{VectorCamp} is a provider of \acrfull{SIMD} Vectorization and Software optimization
  and Training Services.
  \item \textbf{VanTosh} is an \acrfull{ISV} with expertise in porting
  to the \acrshort{POWER} platform, and a \acrfull{MSP} for running
  your private cloud based multi-architecture like \acrshort{POWER} machines, and is
  also very active in open source communities for software and hardware.
\end{itemize}

\subsection{About this project}

The \acrshort{SVP64} Extension is optimised to greatly simplify mathematical calculations,
providing benefits in the fields of Cryptography, Machine Learning,
Autonomous systems, Audio/Video, \acrfull{DSP} and many other general-purpose areas.
 
The building-blocks of \acrshort{SVP64} are designed to provide by combination these
solutions and we are now working, with the support of \acrshort{NGI} Search, to
establish new building-blocks to include Advanced Search capabilities,
which are also applicable to all of the above.

\subsection{What milestones have you set and how are you going about achieving them?}

The three key milestones are: analysis, review and enhancement.
Across the three participating organisations we have a multi-skilled
team and there will be cross-organisational cooperation on all Milestones.

\subsection{What are your goals for the middle/long-term future?}

The ultimate goal is to create a progressive family of micro-processors
and this grant allows to validate the critical elements of the
\acrshort{SVP64} extension, ensuring that the processor family is efficient at string,
data, and machine learning, as wel as provides support for standard open
source software libraries.

\subsection{How is the \acrshort{NGI} Search money helping you?}

At a fundamental level it is buying us ``thinking time".
A considerable amount of time is spent to analyse and understand,
to a purpose. This purpose is the consequences for Search as
opposed to previously funded developments such as \acrshort{NGI} POINTER,
which addressed different goals.

\subsection{How has being part of \acrshort{NGI} Search helped you?}

We have been part of the \acrshort{NGI} Ecosystem now for five years:
NLnet (\acrshort{NGI} Trust and Ensure), \acrshort{NGI} POINTER and now \acrshort{NGI} Search.
Whilst \acrshort{NGI} Search is new and developing we expect it to be as
useful and productive as our previous interactions.

In particular we really appreciate the additional support that
comes with the \acrshort{NGI} Family: the ``added-value" activities, such
as training, sharing of ideas, and Mentoring on strategies for
Business Development.

Now will be a brief introduction on the work planned/done by the three
consortium partners, Red Semiconductor, VectorCamp and VanTosh.

\section{Red Semiconductor}

Red Semiconductor was in charge of planning and organisation of the milestones.
For milestones 2 and 3 onwards, Red focused on documentation and providing
assistance to VectorCamp and VanTosh by writing \acrshort{SVP64} routines and explaining
the features relevant to development within this project.

During Jun 2023, a logo for the project was drawn up.

Interview was done in July 2023 which covered significance of \acrshort{SVP64} and
the importance of funding by \acrshort{NGI} and NLnet.
The interview can be found
\href{https://spaces.fundingbox.com/spaces/ngi-community-ngi-innovators/64b8dadbabf7a659885ee01e}{here}.

Later, an audio recording was made to summarise the goals and progress of the
project. The recording is avalable on the NGI Search website :
\href{https://www.ngisearch.eu/view/Events/OC1Searchers}{here}.

\section{VectorCamp}

VectorCamp is the official maintainer of Vectorscan, a Portable Massively
parallel \acrfull{regex} Matcher library. It is used extensively in
Intrusion Detection Systems software (like Snort, Suricata and others)
and Network Security Analysis in general. Originally the project was forked
from Intel’s Hyperscan but attempts to port it to ARM (originally) were not
accepted upstream. So a fork of Hyperscan was created in the form of
Vectorscan, where portability is the main focus of the project.
Now a few years later, and Vectorscan has become a popular project,
with many external contributions. It is currently heavily developed and
continuously optimized and improved.

At the time of writing it is ported
to ARM and \acrshort{POWER} architectures. Particularly for ARM, Neon/ASIMD support
is 100\% while there is ongoing work to port the code to \acrshort{SVE2} and
the same Fat Runtime functionality as on x86 is implemented,
so that the same binary can run and take advantage of \acrshort{SVE2} if available.
Furthermore, Loongson LSX support is under review and we are in progress of
porting it to even more architectures.

During Milestone 1, VectorCamp worked together with Red and VanTosh to plan out
the project work relating to the use of \acrshort{SVP64} in Vectorscan, and also provided
research and expertise in \acrshort{SIMD} in general.

The primary goal for VectorCamp for \acrshort{NGI} Search is to port
Vectorscan to the Libre-SOC architecture, \acrshort{SVP64} in particular to demonstrate
regex search capabilities.

Due to \acrshort{SVP64} being a Vector architecture,
and not a \acrshort{SIMD} one, and with Vectorscan’s whole codebase designed around \acrshort{SIMD},
this is not such a straightforward task.

The whole codebase is tailored around \acrshort{SIMD} intrinsics and \acrshort{SIMD} data types to
the point that it is currently impossible to run it on an architecture that
lacks a supported \acrshort{SIMD} unit. So, before actual \acrshort{SVP64} development on Vectorscan
can even begin, we have to ensure two things:

\begin{enumerate}
  \item Vectorscan can run on a \acrshort{SIMD}-less architecture, without rewriting
   the whole code base.
  \item Vectorscan has to be adapted -at least partially at first- to make it easy
   for a single algorithm to be ported to \acrshort{SVP64}.
\end{enumerate}

The above tasks are planned to be done during Milestones 2 and 3.

\section{VanTosh}

During Milestone 1, VanTosh contributed to the writing of further
milestone plans alongside Red and VectorCamp. In particular, VanTosh
provided its expertise in C, standard libraries, and software compilation and
maintenance.

Plannig during Milestone 1 resulted in the following tasks for VanTosh:

\begin{enumerate}
  \item Research available C libraries to be used with custom \acrshort{SVP64} routines.
  \item Write several \acrshort{SVP64} routines implementing functionality from those libraries.
\end{enumerate}

The idea of the project is to be as general as possible, which imposes limitations
on the standard C library choice for this project. Full comparison of various
C libraries researched will done and documented during Milestone 2
(see Section \ref{chap:libc}, page \pageref{chap:libc}).

Once the choice a C library has been made, the next goal is to continue working on
several string and memory routines to leverage the capabilities of \acrshort{SVP64},
and demonstrate intergration with a standard test framework.

